\documentclass{pracamgr-en}
\usepackage[T1]{fontenc}
\usepackage{times}
\usepackage[latin2]{inputenc}
\usepackage{comment}
\usepackage{a4wide}
\usepackage{longtable}
\usepackage{multicol} 
\usepackage{semantic}
\usepackage{casl}
\usepackage{color} 
\usepackage{url}
\usepackage[algoruled,vlined]{algorithm2e}
\usepackage{float}
\usepackage{latexsym}

\input{xy}
\xyoption{all}
\input{diagxy}

% Dane magistranta:

\author{Maciej Makowski}

\nralbumu{189188}

\kierunek{Computer\ \ Science}

% Praca wykonana pod kierunkiem:
% (poda� tytu�/stopie� imi� i nazwisko opiekuna
% Instytut 
% ew. Wydzia� ew. Uczelnia (je�eli nie MIM UW))
\opiekun{prof. dr hab. Andrzej Tarlecki\\
  Institute of Informatics\\
  }

% miesi�c i~rok:
\date{August 2004}

% Poda� klasyfikacj� tematyczn� wed�ug AMS MSC 2000 lub ACM CCS:
\klasyfikacja{D.2.1 Requirement/Specification\\
  D.2.2 Design Tools and Techniques\\
  F.3.1 Specifying and Verifying and Reasoning about Programs}

% Tu jest dobre miejsce na Twoje w�asne makra i~�rodowiska:
% \newtheorem{defi}{Definicja}[section]
\def\TODO{\\\\ \begin{center}{\bf TODO}\end{center}\\\\\\}
\def\CASL{\textsc{Casl}}
\def\HasCASL{\textsc{HasCasl}}
\def\CspCASL{\textsc{Csp-Casl}}
\def\Hets{\textsc{Hets}}
\def\webaddr #1{\url{#1}}
\def\cat #1{\mathrm {\bf #1}}
\def\Sign{\cat{Sign}}
\def\Mod{\cat{Mod}}
\def\Sen{\cat{Sen}}
\def\code #1{\(#1\)}
\def\nterm #1{{\sl #1}}
\def\module #1{\texttt{#1}}
\def\path #1{\texttt{#1}}
\def\morphpath #1{\langle #1\rangle}
\def\seq #1{\langle #1\rangle}
\def\rulename #1{{\bf(#1)}}
\def\return #1{\KwSty{return} #1}
\def\algline #1{{\footnotesize{\bf #1}}}
\def\centercmd #1{\begin{center}{\tt #1}\end{center}}
\def\sym #1{$\mathit{#1}$}
\def\msym #1{\mathit{#1}}
\def\NodesD{\mathit{Nodes}(D)}
\def\EdgesD{\mathit{Edges}(D)}
\newtheorem{stdth}{Theorem}
\newcommand{\proof}{\par \vskip 5mm\noindent{\bf Proof\@: }}
\newcommand{\QED}{{\unskip\nobreak\hfil\penalty50
  \hskip2em\hbox{}\nobreak\hfil$\Box$
  \parfillskip=0pt \finalhyphendemerits=0 \par}}
\renewcommand{\QED}{{\unskip\nobreak\hfil\penalty50
  \hskip2em\hbox{}\nobreak\hfill $\Box$
  \parfillskip=0pt \finalhyphendemerits=0 \vspace*{3ex}\\}}
\newcommand{\QUERY}[1]%{}
{\marginpar{\raggedright\hspace{0pt}\small #1\\~}}

\newcommand{\eat}[1]{}

\newenvironment{EXAMPLE}[1][]   {\par#1\begin{EXAMPLEFORMAT}\begin{ITEMS}}
                                {\end{ITEMS}\end{EXAMPLEFORMAT}}
\newcommand{\IEXT}[1]           {\\#1\I}
\newcommand{\IEND}              {\I\END}
\newenvironment{EXAMPLEFORMAT}  {}{}

%% Added by MB to have some extra vertical space after the ``main'' examples
%% following the points (and some others in the text): 
\newenvironment{BIGEXAMPLE}   {\begin{EXAMPLE}} {\end{EXAMPLE}\medskip}
\newenvironment{DETAILS}[1][]   {#1\begin{DETAILSFORMAT}}{\end{DETAILSFORMAT}}
\newenvironment{DETAILSFORMAT}  {}{}
\newenvironment{META}[1][]      {#1\begin{METAFORMAT}}{\end{METAFORMAT}}
\newenvironment{METAFORMAT}     {\medskip\vrule\hspace{1ex}\vrule\hspace{1ex}%
                                 \begin{minipage}{0.9\textwidth}\it}
                                {\end{minipage}\par\medskip}

\newcommand{\SLIDESMALL}        {}
\newcommand{\SLIDESONLY}[1]     {}


%%%%%%%%%%%%%%%%%%%%%%%%%%%%%%%%%%%%%%%%%%%%%%%%%%%%%%%%%%%%%%%%%%%%%%
% SIMULATING SMALL-CAPS FOR BOLD, EMPH

\newcommand{\normalTEXTSC}[2]{{#1\scriptsize#2}}
%% NOT \newcommand{\normalTEXTSC}[2]{{\normalsize#1\scriptsize#2}}
\newcommand{\largeTEXTSC} [2]{{\large     #1\small     #2}}
\newcommand{\LargeTEXTSC} [2]{{\Large     #1\normalsize#2}}
\newcommand{\LARGETEXTSC} [2]{{\LARGE     #1\large     #2}}
\newcommand{\hugeTEXTSC}  [2]{{\huge      #1\Large     #2}}
\newcommand{\HugeTEXTSC}  [2]{{\Huge      #1\LARGE     #2}}


%\newcommand     {\CASL}{\normalTEXTSC{C}{ASL}\xspace}
\newcommand{\largeCASL} {\largeTEXTSC{C}{ASL}\xspace}
\newcommand{\LargeCASL} {\LargeTEXTSC{C}{ASL}\xspace}
\newcommand{\LARGECASL} {\LARGETEXTSC{C}{ASL}\xspace}
\newcommand {\hugeCASL}  {\hugeTEXTSC{C}{ASL}\xspace}
\newcommand {\HugeCASL}  {\HugeTEXTSC{C}{ASL}\xspace}

%\newcommand     {\CoFI}{CoFI\xspace}

\newcommand     {\MAYA}{\normalTEXTSC{M}{AYA}\xspace}
\newcommand{\largeMAYA} {\largeTEXTSC{M}{AYA}\xspace}

\newcommand     {\Hets}{\normalTEXTSC{H}{ETS}\xspace}
\newcommand{\largeHets} {\largeTEXTSC{H}{ETS}\xspace}
\newcommand{\LARGEHets} {\LARGETEXTSC{H}{ETS}\xspace}

\newcommand     {\Cats}{\normalTEXTSC{C}{ATS}\xspace}
\newcommand{\largeCats} {\largeTEXTSC{C}{ATS}\xspace}

\newcommand     {\ELAN}{\normalTEXTSC{E}{LAN}\xspace}
\newcommand{\largeELAN} {\largeTEXTSC{E}{LAN}\xspace}

\newcommand     {\HOL}{\normalTEXTSC{H}{OL}\xspace}
\newcommand{\largeHOL} {\largeTEXTSC{H}{OL}\xspace}

\newcommand     {\Isabelle}{\normalTEXTSC{I}{SABELLE}\xspace}
\newcommand{\largeIsabelle} {\largeTEXTSC{I}{SABELLE}\xspace}

\newcommand     {\Horn}{\normalTEXTSC{H}{ORN}}

%% Use \ELAN-\CASL, \HOL-\CASL, \Isabelle/\HOL

\newcommand{\LCF}{LCF\xspace}

\newcommand{\ASF}{ASF\xspace}
%%\newcommand     {\ASF}{\normalTEXTSC{A}{SF}\xspace}
%%\newcommand{\largeASF} {\largeTEXTSC{A}{SF}\xspace}

\newcommand{\SDF}{SDF\xspace}
%%\newcommand     {\SDF}{\normalTEXTSC{S}{DF}\xspace}
%%\newcommand{\largeSDF} {\largeTEXTSC{S}{DF}\xspace}

\newcommand     {\ASFSDF}{\normalTEXTSC{A}{SF}+\normalTEXTSC{S}{DF}\xspace}
\newcommand{\largeASFSDF} {\largeTEXTSC{A}{SF}+\largeTEXTSC{S}{DF}\xspace}

\newcommand     {\HasCASL}{\normalTEXTSC{H}{AS}\normalTEXTSC{C}{ASL}\xspace}
\newcommand{\largeHasCASL} {\largeTEXTSC{H}{AS}\largeTEXTSC{C}{ASL}\xspace}

%% Do NOT use \ASF+\SDF (it gives a superfluous space in the middle)

\newcommand{\CCC}{CCC\xspace}

\newcommand{\CoCASL}{\normalTEXTSC{C}{O}\normalTEXTSC{C}{ASL}\xspace}
\newcommand{\CspCASL}{\normalTEXTSC{C}{SP}-\normalTEXTSC{C}{ASL}\xspace}
\newcommand{\Csp}{\normalTEXTSC{C}{SP}\xspace}
\newcommand{\CcsCASL}{CCS-\normalTEXTSC{C}{ASL}\xspace}
\newcommand{\CASLLtl}{\normalTEXTSC{C}{ASL}-\normalTEXTSC{L}{TL}\xspace}
\newcommand{\CASLChart}{\normalTEXTSC{C}{ASL}-\normalTEXTSC{C}{HART}\xspace}
\newcommand{\SBCASL}{\normalTEXTSC{S}{B}-\normalTEXTSC{C}{ASL}\xspace}
\newcommand{\HetCASL}{\normalTEXTSC{H}{ET}\normalTEXTSC{C}{ASL}\xspace}



% koniec definicji

\title{An Implementation of a Tool for \CASL\ Architectural Specification Analysis}


% S�owa kluczowe:
\keywords{\CASL, architectural specifications, static
  analysis, amalgamability}

\begin{document}
\maketitle

\begin{abstract}
The static analysis of
architectural specifications in \CASL\ and the way it has been
implemented by the author in \nterm{Heterogeneous \CASL\ Tool Set}
(\Hets) is described. All the 
stages of processing the specification are presented. The algorithm
for checking amalgamability conditions in \CASL\ institution is
discussed in detail.
\end{abstract}

\tableofcontents
\listoffigures
\listofalgorithms
%\listoftables


% ====================================================================
% INTRODUCTION

\chapter{Introduction}
%\addcontentsline{toc}{chapter}{Introduction}

This thesis describes the static analysis of \CASL\
architecural specifications as it is implemented in \Hets\
system. The current chapter serves as a brief introduction to \CASL\
and an 
overview of \Hets. However, reading a thorough description of
\CASL\ (e.g. \cite{userman}) is strongly recommended and may be
necessary for understanding more advanced \CASL\ features.

Chapter \ref{staticAnalysis} describes the formal definition of \CASL\
semantics and the way it may be used to verify static correctness of
architectural specifications. The notion of \nterm{institution} is
introduced and \CASL-specific amalgamability checking problem
is presented.
The natural formulations of amalgamability condition and cell calculus
rely on category theory. For the sake of clarity and brevity more
advanced category-theoretical interpretations have been ommitted and
intuitive, set-theoretical interpretations are provided instead.
The assumption was that the ease of comprehension should take
precedence over formalism; the descriptions are quite verbose in
order to make 
understanding the concepts presented easier. For strictly formal
definition of the cell calculus and cell condition please refer to
\cite{khtsm1}. Anyway, familiarity with basic category theory may be
necessary for understanding the concepts presented in this chapter.

This thesis is based on implementation work done by the author, which
includes parsing of architectural specifications, static analysis of
architectural specifications and amalgamation analysis for \CASL. In
chapter \ref{impl} the implementation details are described. Each 
stage of processing the specification is treated separately, with the
data structures and functions specified. Whenever a non-trivial
algorithm is applied, it is illustrated using pseudo-code. The
knowledge of \nterm{Haskell} programming language might be helpful,
but is not required.

Similar implementation (completed as a part of CATS system) has
already been described by Bartek 
Klin in \cite{klin1}. The implementation described here uses updated
static semantics (defined in \cite{refman}) and more recent results
regarding 
amalgamability condition (presented in \cite{khtsm1}). 

\section{\CASL}

{\sl Algebraic specification} is one of the most extensively developed
approaches in the area of formal methods. In this approach the program
is modelled as a {\sl many-sorted algebra} with sets, values and
functions over those sets. A great deal of research on the theory and
practice of algebraic specifications has been
done for past 25 years. However, the proliferation of different
algebraic specification languages rendered creating a common library
of examples, case studies etc. impossible and therefore has been a
major obstacle to adopting these techniques in the industrial
context. The need for a common framework resulted in 1995 in the
initiation of {\sl The Common Framework Initiative}
(CoFI)\footnote{CoFI is pronounced like 'coffee'. See
  \webaddr{http://www.cofi.info}.}. As soon as in 1997 
an initial design (with language summary, abstract syntax and formal
semantics but lacking concrete syntax) of {\it Common Algebraic
  Specification Language} (\CASL) was proposed. The version 1.0 of
\CASL\ was released in October 1998, version 1.0.1 followed in April
2001. The current version, 1.0.2, is documented in the \CASL\
Reference Manual \cite{refman}. 

\CASL\ is composed of four layers; from bottom to top these are:
\begin{itemize}
\item \nterm{basic specifications} --- contain sorts, operations,
  predicates and axioms which constrain the behaviour of operations
  and predicates;
\item \nterm{structured specifications} --- are formed from basic
  specifications, references to named specifications and
  instantiations of generic specifications;
\item \nterm{architectural specifications} --- allow to specify 
  architecture of the software to be implemented;
\item \nterm{specification libraries} --- group the specifications from
  previous layers.
\end{itemize}


\subsection{Basic Specifications}

A \nterm{basic specification} consists of \nterm{symbol declarations} and
\nterm{axioms} and \nterm{constraints} that restrict the possible
\nterm{interpretations} of these symbols. The declared symbols may
represent sorts, operations or predicates interpreted in models in the
standard way. A subsorting relation
between sorts might be specified; if $s$ is declared to be a subsort
of $t$ (written $s < t$) this is interpreted not as an inclusion of the
carrier set of $s$ into the carrier set of $t$ but more generally, as
an \nterm{embedding}, i.e. a 1-1 function from the carrier set of $s$
to the carrier set of $t$. The subsort embeddings are required to
commute: subsort declarations $s < t$, $t < u$ and $s < u$ specify the
existence of embeddings $\alpha$, $\beta$ and $\gamma$ (respectively);
it is then required that $\beta\circ\alpha = \gamma$. Moreover, the
embedding from $s$ to $s$ is required to be identity. These subsort
embedding compatibility requirements play important role when the
possibility of combining units is introduced (\ref{casl:archSpec}).
\begin{BIGEXAMPLE}
\I\SPEC \NAMEREF{Strict\_Partial\_Order} =
%%PDM\I{}    \COMMENTENDLINE{Let's start with a simple example !}
\begin{ITEMS}[\PRED]
\I\SORT    \( Elem \) 
\I\PRED    \( \_\_<\_\_ : Elem \* Elem \)
%           \COMMENTENDLINE{\PRED abbreviates predicate}
\end{ITEMS}
\(\[  \FORALL x,y,z : Elem \\
      \. \NOT(x < x)                      \RIGHT{\LABEL{strict}}     \\
      \. x < y   \IMPLIES  \NOT(y < x)    \RIGHT{\LABEL{asymmetric}} \\
      \. x < y \A y < z  \IMPLIES  x < z  \RIGHT{\LABEL{transitive}} \\
\]\)
\begin{COMMENT}
Note that there may exist \(x, y\) such that\\
neither \(x < y\) nor \(y < x\).
\end{COMMENT}
\I\END
\end{BIGEXAMPLE}
The specification above declares one sort named $Elem$ and a predicate
$<$\footnote{Not to be confused with the subsort embedding.}, whose
intended interpretation is specified with the axioms that 
follow the declarations. The \nterm{semantics} of a basic specification
consists of: 
\begin{itemize}
\item a \nterm{signature} $\Sigma$ containing the defined symbols and
\item a \nterm{class of $\Sigma$-models} containing those
  interpretations of the signature that satisfy the axioms.
\end{itemize}


\subsection{Structured Specifications}
Structured specifications are formed from basic specifications,
references to named specifications and instantiations of generic
specifications using following constructs:
\begin{itemize}
\item \nterm{translation} --- renames the symbols defined in a
  specification;
\item \nterm{hiding} --- removes symbols from a specification;
\item \nterm{union} --- composes the specifications identifying common
  symbols;
\item \nterm{extension} --- adds further symbols and axioms to a
  specification;
\item \nterm{free extension} --- adds further symbols and axioms and
  restricts the models of resulting specification to free models; if
  the specification extended is empty, then the models of free
  extension are just initial models;
\item \nterm{generic specification} --- defines specification
  parameters for which arguments have to be provided whenever such
  specification is referenced.
\end{itemize}
\begin{BIGEXAMPLE}
\I\SPEC \NAMEREF{Generic\_Monoid} [ \SORT $Elem$ ] =
\begin{ITEMS}[\PRED]
\I\SORT    \( Monoid \) 
\I\OPS     $inj : Elem \rightarrow Monoid$\\
           $1:Monoid$\\
           $\_\_*\_\_ : Monoid \* Monoid\rightarrow
Monoid,\,assoc,\,unit\,1$
\end{ITEMS}
\(  \FORALL x,y : Elem \. inj(x) = inj(y) \IMPLIES x = y \)
\I\END
\end{BIGEXAMPLE}
\begin{BIGEXAMPLE}
\I\SPEC \NAMEREF{Generic\_Commutative\_Monoid} [ \SORT $Elem$ ] =
\I\NAMEREF{Generic\_Monoid} [ \SORT $Elem$ ]
\I\I\THEN \(  \FORALL x,y : Monoid \. x * y = y * x \)
\I\END
\end{BIGEXAMPLE}
\NAMEREF{Generic\_Monoid} is a generic specification parameterised by
sort $Elem$. \NAMEREF{Generic\_Commutative\_Monoid} is a (generic)
extension of \NAMEREF{Generic\_Monoid} that requires the monoid
operation to be commutative. The semantics of structured
specifications is of the same kind as that of basic specifications.


\subsection{Architectural Specifications}\label{casl:archSpec}
Whilst structured specification constructs allow one to write
specifications in modular fashion, they are merely syntactical
operations and do not impose any structure on models. In order to
specify the way in 
which software modules should be constructed and composed during the
implementation, architectural specifications are introduced.
\begin{BIGEXAMPLE}
\I\ARCHSPEC \NAMEDEFN{System\_1} =
\begin{ITEMS}
\I\UNITS \[
  Ord : \NAMEDEFN{Strict\_Partial\_Order};\\
  Col : \NAMEDEFN{Color};\\
  Cont : \NAMEDEFN{Elem} \TOTAL \NAMEDEFN{Container[Elem]};
\]
\I\RESULT Cont[Ord] \AND Cont[Col \FIT $Elem$ \mapsto \(RGB\)]
\end{ITEMS}
\I\END 
\end{BIGEXAMPLE} 
Assuming that we have previously defined specifications
\NAMEDEFN{Strict\_Partial\_Order} (with sort $Elem$), \NAMEDEFN{Color} (with
sort \(RGB\)) and a generic specification \NAMEDEFN{Container}[\SORT
  $Elem$] the architecture \NAMEDEFN{System\_1} specified above
requires its model to have defined:
\begin{itemize}
\item a module Ord satisfying (implementing) the specification
  \NAMEDEFN{Strict\_Partial\_Order}, 
\item a module Col satisfying the specification \NAMEDEFN{Color},
\item a parameterised module Cont that, given a module with sort $Elem$,
  produces a container module for that sort.
\end{itemize}
Finally the \RESULT section describes the way those modules are
composed in order to produce the complete system: container modules for Ord
and Col are constructed (the Cont parameterised module expects sort
$Elem$, so a mapping has to be provided saying that \(RGB\) should be
the sort of data stored in resulting containers) and are amalgamated to
provide one module with the functionality of both containers. 

The semantics of an architectural specification reflects its
structure. The component units (modules) may be regarded as unit functions
(functions without arguments yield self-conained units); the
entire specification is a collection of those functions together with
the result of their composition according to the description in
\RESULT.

The point that requires special attention is handling amalgamation of
two units into one. The semantics requires that whenever such
amalgamation takes place the amalgamated units must share common
components (i.e.~the interpretations of symbols from the intersection of
their signatures). Moreover, subsort embeddings must also be
compatible (for instance, implicit embeddings, resulting from
composition of embeddings from the amalgamated units, must be defined
unambiguously) in order for the amalgamation to exist. The
amalgamation 
conditions should be checked statically, as is the case e.g.~with
sharing conditions in Standard ML. Amalgamation condition checking will
be the main problem dealt with in this thesis.


\section{\Hets}


In order to practically apply \CASL\ specification techniques to
larger projects, tools supporting \CASL\ are being developed. The {\sl
  \CASL\ 
Tool Set} (CATS) developed in Bremen\footnote{See
  \webaddr{http://www.informatik.uni-bremen.de/cofi/Tools/CATS.html}.}
using Standard ML supports parsing, static analysis, mixfix analysis,
encoding for theorem proving tools and \LaTeX\ formatting of \CASL\
specifications. However, the \CASL\ logic is not always the most
convenient formalism to specify system modules, sometimes it would be
better to use a domain-specific logic. Hence a desired feature would
be support for \nterm{heterogeneous \CASL} --- a superset of \CASL\
that allows mixing different logics in the specifications. 

CATS has been designed with genericity in mind: static analysis of
structured specifications has been implemented as a Standard ML
functor parameterised over static analysis for arbitrary
logic. Therefore extending the tool set to support new logic boils
down to providing a tool for analysis of the specifications
in-the-small (i.e. basic specifications)
and instantiating the structured specification analysis with that
tool. However, this approach presents serious limitations when
heterogeneity is to be achieved: a separate instance of static
analysis module would have to be created for each logic used in
specification being analysed and it would not be possible to deal with these
instances in an uniform way, especially when the number of required
instances is not known in advance.

Therefore a decision has been made to move to Haskell,
whose type system is more expressive; type system
extensions provided by the Glasgow Haskell Compiler (GHC) are particularly
useful for handling the heterogeneity in an elegant way. Present work
in Bremen focuses on the \nterm{Heteregeneous Tool Set} (\Hets)\footnote{See
  \webaddr{http://www.informatik.uni-bremen.de/agbkb/forschung/formal_methods/CoFI/hets/}.} 
\cite{moss1} --- a set of tools implemented in Haskell that supports
the analysis of heterogeneous \CASL\ specifications.

\subsection{Heterogeneous Specifications}

In \Hets\ the default logic for analysed library is determined based on the
input file extension; e.g. \path{.casl} indicates that \CASL\ logic
should be used by default. It is possible to change the current logic
within the specification using \texttt{logic {\sl logic\_name}}
construct, e.g.
\begin{verbatim}
logic HasCASL
\end{verbatim}
The specifications that follow are then treated as \HasCASL\
specifications. The example below shows how heterogeneity can be
exploited to specify a data type using \CASL\ and process that involves
this data type using \CspCASL. 
\begin{verbatim}
library Buffer

logic CASL

spec List =
  free type List[Elem] ::= nil | cons(Elem; List[Elem])
end

logic CspCASL

spec Buffer =
  data List
  channel read, write : Elem
  process read   
  let Buf(l : List[Elem]) =
              read ? x -> Buf(cons(x, nil))
              [] if l = nil then STOP 
                            else write ! last(l) -> Buf(rest(l))
  in Buf(nil)
end 
\end{verbatim}

\subsection{Architecture of \Hets}

\begin{figure}[h]
\begin{center}
\vspace{1em}
\setlength{\unitlength}{1740sp}%
%
\begingroup\makeatletter\ifx\SetFigFont\undefined%
\gdef\SetFigFont#1#2#3#4#5{%
  \reset@font\fontsize{#1}{#2pt}%
  \fontfamily{#3}\fontseries{#4}\fontshape{#5}%
  \selectfont}%
\fi\endgroup%
\begin{picture}(12534,8879)(439,-8907)
\thinlines
{\color[rgb]{0,0,0}\put(2458,-5596){\vector( 0,-1){630}}
}%
{\color[rgb]{0,0,0}\put(2458,-4799){\vector( 0,-1){270}}
}%
{\color[rgb]{0,0,0}\put(2458,-4269){\line( 0,-1){260}}
}%
{\color[rgb]{0,0,0}\put(2458,-3645){\vector( 0,-1){270}}
}%
{\color[rgb]{0,0,0}\put(2458,-3115){\line( 0,-1){260}}
}%
{\color[rgb]{0,0,0}\put(2458,-2491){\vector( 0,-1){270}}
}%
{\color[rgb]{0,0,0}\put(2458,-1961){\line( 0,-1){260}}
}%
{\color[rgb]{0,0,0}\put(10773,-5551){\vector( 0,-1){900}}
}%
{\color[rgb]{0,0,0}\put(10773,-4799){\vector( 0,-1){270}}
}%
{\color[rgb]{0,0,0}\put(10773,-4269){\line( 0,-1){260}}
}%
{\color[rgb]{0,0,0}\put(10773,-3645){\vector( 0,-1){270}}
}%
{\color[rgb]{0,0,0}\put(10773,-3115){\line( 0,-1){260}}
}%
{\color[rgb]{0,0,0}\put(10773,-2491){\vector( 0,-1){270}}
}%
{\color[rgb]{0,0,0}\put(10773,-1961){\line( 0,-1){260}}
}%
{\color[rgb]{0,0,0}\put(6436,-3391){\line(-2, 3){595.385}}
}%
{\color[rgb]{0,0,0}\put(6571,-3391){\line( 0, 1){1530}}
}%
{\color[rgb]{0,0,0}\put(6706,-3391){\line( 2, 3){595.385}}
}%
{\color[rgb]{0,0,0}\put(7039,-3523){\line( 3, 2){488.077}}
}%
{\color[rgb]{0,0,0}\put(6226,-3548){\line(-3, 2){488.077}}
}%
{\color[rgb]{0,0,0}\put(5791,-4046){\line( 3, 2){405}}
}%
{\color[rgb]{0,0,0}\put(7253,-4056){\line(-3, 2){405}}
}%
{\color[rgb]{0,0,0}\put(6795,-4557){\line( 5, 2){411.207}}
}%
{\color[rgb]{0,0,0}\put(6283,-4558){\line(-5, 2){411.207}}
}%
{\color[rgb]{0,0,0}\put(5691,-5146){\line( 3, 2){405}}
}%
{\color[rgb]{0,0,0}\put(7403,-5146){\line(-3, 2){405}}
}%
{\color[rgb]{0,0,0}\put(6571,-5056){\line( 0, 1){180}}
}%
{\color[rgb]{0,0,0}\put(10773,-2755){\oval(3785,5632)}
%\put(9106,-164){\oval(210,210)[tl]}
%\put(12541,-5446){\oval(210,210)[br]}
%\put(12541,-164){\oval(210,210)[tr]}
%\put(9106,-5551){\line( 1, 0){3435}}
%\put(9106,-59){\line( 1, 0){3435}}
%\put(9001,-5446){\line( 0, 1){5282}}
%\put(12646,-5446){\line( 0, 1){5282}}
}%
{\color[rgb]{0,0,0}\put(2458,-2755){\oval(3785,5632)}
%{\color[rgb]{0,0,0}\put(781,-5491){\oval(210,210)[bl]}
%\put(781,-145){\oval(210,210)[tl]}
%\put(4306,-5491){\oval(210,210)[br]}
%\put(4306,-145){\oval(210,210)[tr]}
%\put(781,-5596){\line( 1, 0){3525}}
%\put(781,-40){\line( 1, 0){3525}}
%\put(676,-5491){\line( 0, 1){5346}}
%\put(4411,-5491){\line( 0, 1){5346}}
}%
{\color[rgb]{0,0,0}\put(6623,-2755){\oval(3785,5632)}
%{\color[rgb]{0,0,0}\put(4966,-5491){\oval(210,210)[bl]}
%\put(4966,-166){\oval(210,210)[tl]}
%\put(8430,-5491){\oval(210,210)[br]}
%\put(8430,-166){\oval(210,210)[tr]}
%\put(4966,-5596){\line( 1, 0){3464}}
%\put(4966,-61){\line( 1, 0){3464}}
%\put(4861,-5491){\line( 0, 1){5325}}
%\put(8535,-5491){\line( 0, 1){5325}}
}%
{\color[rgb]{0,0,0}\put(4361,-2851){\line( 1, 0){350}}
}%
{\color[rgb]{0,0,0}\put(8521,-2851){\vector( 1, 0){350}}
}%
{\color[rgb]{0,0,0}\put(751,-7276){\oval(330,330)[bl]}
\put(751,-6526){\oval(330,330)[tl]}
\put(2941,-7276){\oval(330,330)[br]}
\put(2941,-6526){\oval(330,330)[tr]}
\put(751,-7441){\line( 1, 0){2190}}
\put(751,-6361){\line( 1, 0){2190}}
\put(586,-7276){\line( 0, 1){750}}
\put(3106,-7276){\line( 0, 1){750}}
}%
{\color[rgb]{0,0,0}\put(586,-8760){\oval(270,270)[bl]}
\put(586,-6361){\oval(270,270)[tl]}
\put(5851,-8760){\oval(270,270)[br]}
\put(5851,-6361){\oval(270,270)[tr]}
\put(586,-8895){\line( 1, 0){5265}}
\put(586,-6226){\line( 1, 0){5265}}
\put(451,-8760){\line( 0, 1){2399}}
\put(5986,-8760){\line( 0, 1){2399}}
}%
{\color[rgb]{0,0,0}\put(1726,-8551){\oval(300,300)[bl]}
\put(1726,-7906){\oval(300,300)[tl]}
\put(4411,-8551){\oval(300,300)[br]}
\put(4411,-7906){\oval(300,300)[tr]}
\put(1726,-8701){\line( 1, 0){2685}}
\put(1726,-7756){\line( 1, 0){2685}}
\put(1576,-8551){\line( 0, 1){645}}
\put(4561,-8551){\line( 0, 1){645}}
}%
{\color[rgb]{0,0,0}\put(8537,-8101){\oval(300,300)[bl]}
\put(8537,-6646){\oval(300,300)[tl]}
\put(12811,-8101){\oval(300,300)[br]}
\put(12811,-6646){\oval(300,300)[tr]}
\put(8537,-8251){\line( 1, 0){4274}}
\put(8537,-6496){\line( 1, 0){4274}}
\put(8387,-8101){\line( 0, 1){1455}}
\put(12961,-8101){\line( 0, 1){1455}}
}%
{\color[rgb]{0,0,0}\put(3481,-7290){\oval(300,300)[bl]}
\put(3481,-6531){\oval(300,300)[tl]}
\put(5656,-7290){\oval(300,300)[br]}
\put(5656,-6531){\oval(300,300)[tr]}
\put(3481,-7440){\line( 1, 0){2175}}
\put(3481,-6381){\line( 1, 0){2175}}
\put(3331,-7290){\line( 0, 1){759}}
\put(5806,-7290){\line( 0, 1){759}}
}%
{\color[rgb]{0,0,0}\put(5986,-7441){\vector( 1, 0){2385}}
}%
\put(2135,-1861){\makebox(0,0)[lb]{\smash{\SetFigFont{7}{8.4}{\sfdefault}{\mddefault}{\updefault}{\color[rgb]{0,0,0}Text}%
}}}
\put(2106,-2438){\makebox(0,0)[lb]{\smash{\SetFigFont{7}{8.4}{\sfdefault}{\mddefault}{\updefault}{\color[rgb]{0,0,0}\emph{Parser}}%
}}}
\put(1441,-3015){\makebox(0,0)[lb]{\smash{\SetFigFont{7}{8.4}{\sfdefault}{\mddefault}{\updefault}{\color[rgb]{0,0,0}Abstract syntax}%
}}}
\put(1616,-3592){\makebox(0,0)[lb]{\smash{\SetFigFont{7}{8.4}{\sfdefault}{\mddefault}{\updefault}{\color[rgb]{0,0,0}\emph{Static analysis}}%
}}}
\put(1131,-4169){\makebox(0,0)[lb]{\smash{\SetFigFont{7}{8.4}{\sfdefault}{\mddefault}{\updefault}{\color[rgb]{0,0,0}(Signature, Sentences)}%
}}}
\put(1896,-4746){\makebox(0,0)[lb]{\smash{\SetFigFont{7}{8.4}{\sfdefault}{\mddefault}{\updefault}{\color[rgb]{0,0,0}\emph{Interfaces}}%
}}}
\put(1641,-5323){\makebox(0,0)[lb]{\smash{\SetFigFont{7}{8.4}{\sfdefault}{\mddefault}{\updefault}{\color[rgb]{0,0,0}XML, ATerms}%
}}}
\put(6301,-3706){\makebox(0,0)[lb]{\smash{\SetFigFont{7}{8.4}{\sfdefault}{\mddefault}{\updefault}{\color[rgb]{0,0,0}CASL}%
}}}
\put(5356,-2356){\makebox(0,0)[lb]{\smash{\SetFigFont{7}{8.4}{\sfdefault}{\mddefault}{\updefault}{\color[rgb]{0,0,0}CoCASL}%
}}}
\put(6976,-2356){\makebox(0,0)[lb]{\smash{\SetFigFont{7}{8.4}{\sfdefault}{\mddefault}{\updefault}{\color[rgb]{0,0,0}CASL-LTL}%
}}}
\put(6121,-1726){\makebox(0,0)[lb]{\smash{\SetFigFont{7}{8.4}{\sfdefault}{\mddefault}{\updefault}{\color[rgb]{0,0,0}CSP-CASL}%
}}}
\put(7201,-3121){\makebox(0,0)[lb]{\smash{\SetFigFont{7}{8.4}{\sfdefault}{\mddefault}{\updefault}{\color[rgb]{0,0,0}SB-CASL}%
}}}
\put(5041,-3166){\makebox(0,0)[lb]{\smash{\SetFigFont{7}{8.4}{\sfdefault}{\mddefault}{\updefault}{\color[rgb]{0,0,0}HasCASL}%
}}}
\put(5086,-4336){\makebox(0,0)[lb]{\smash{\SetFigFont{7}{8.4}{\sfdefault}{\mddefault}{\updefault}{\color[rgb]{0,0,0}SubFOL$^=$}%
}}}
\put(7246,-4336){\makebox(0,0)[lb]{\smash{\SetFigFont{7}{8.4}{\sfdefault}{\mddefault}{\updefault}{\color[rgb]{0,0,0}PFOL$^=$}%
}}}
\put(6346,-4786){\makebox(0,0)[lb]{\smash{\SetFigFont{7}{8.4}{\sfdefault}{\mddefault}{\updefault}{\color[rgb]{0,0,0}FOL$^=$}%
}}}
\put(6346,-5371){\makebox(0,0)[lb]{\smash{\SetFigFont{7}{8.4}{\sfdefault}{\mddefault}{\updefault}{\color[rgb]{0,0,0}Horn$^=$}%
}}}
\put(5400,-5371){\makebox(0,0)[lb]{\smash{\SetFigFont{7}{8.4}{\sfdefault}{\mddefault}{\updefault}{\color[rgb]{0,0,0}$\bullet$}%
}}}
\put(7600,-5371){\makebox(0,0)[lb]{\smash{\SetFigFont{7}{8.4}{\sfdefault}{\mddefault}{\updefault}{\color[rgb]{0,0,0}$\bullet$}%
}}}
\put(1006,-421){\makebox(0,0)[lb]{\smash{\SetFigFont{7}{8.4}{\rmdefault}{\bfdefault}{\updefault}{\color[rgb]{0,0,0}Basic specifications}%
}}}
\put(826,-826){\makebox(0,0)[lb]{\smash{\SetFigFont{7}{8.4}{\rmdefault}{\bfdefault}{\updefault}{\color[rgb]{0,0,0}(logic-specific tools for}%
}}}
\put(826,-1231){\makebox(0,0)[lb]{\smash{\SetFigFont{7}{8.4}{\rmdefault}{\bfdefault}{\updefault}{\color[rgb]{0,0,0}CASL and extensions)}%
}}}
\put(5411,-421){\makebox(0,0)[lb]{\smash{\SetFigFont{7}{8.4}{\rmdefault}{\bfdefault}{\updefault}{\color[rgb]{0,0,0}Graph of CASL}%
}}}
\put(9591,-421){\makebox(0,0)[lb]{\smash{\SetFigFont{7}{8.4}{\rmdefault}{\bfdefault}{\updefault}{\color[rgb]{0,0,0}Structured and}%
}}}
\put(9771,-826){\makebox(0,0)[lb]{\smash{\SetFigFont{7}{8.4}{\rmdefault}{\bfdefault}{\updefault}{\color[rgb]{0,0,0}architectural}%
}}}
\put(9796,-1231){\makebox(0,0)[lb]{\smash{\SetFigFont{7}{8.4}{\rmdefault}{\bfdefault}{\updefault}{\color[rgb]{0,0,0}specifications}%
}}}
\put(10501,-1861){\makebox(0,0)[lb]{\smash{\SetFigFont{7}{8.4}{\sfdefault}{\mddefault}{\updefault}{\color[rgb]{0,0,0}Text}%
}}}
\put(10441,-2428){\makebox(0,0)[lb]{\smash{\SetFigFont{7}{8.4}{\sfdefault}{\mddefault}{\updefault}{\color[rgb]{0,0,0}\emph{Parser}}%
}}}
\put(9886,-3015){\makebox(0,0)[lb]{\smash{\SetFigFont{7}{8.4}{\sfdefault}{\mddefault}{\updefault}{\color[rgb]{0,0,0}Abstract syntax}%
}}}
\put(9936,-3592){\makebox(0,0)[lb]{\smash{\SetFigFont{7}{8.4}{\sfdefault}{\mddefault}{\updefault}{\color[rgb]{0,0,0}\emph{Static analysis}}%
}}}
\put(9641,-4169){\makebox(0,0)[lb]{\smash{\SetFigFont{7}{8.4}{\sfdefault}{\mddefault}{\updefault}{\color[rgb]{0,0,0}Development graph}%
}}}
\put(10216,-4746){\makebox(0,0)[lb]{\smash{\SetFigFont{7}{8.4}{\sfdefault}{\mddefault}{\updefault}{\color[rgb]{0,0,0}\emph{Interfaces}}%
}}}
\put(9946,-5323){\makebox(0,0)[lb]{\smash{\SetFigFont{7}{8.4}{\sfdefault}{\mddefault}{\updefault}{\color[rgb]{0,0,0}XML, ATerms}%
}}}
\put(2476,-8476){\makebox(0,0)[lb]{\smash{\SetFigFont{7}{8.4}{\sfdefault}{\mddefault}{\updefault}{\color[rgb]{0,0,0}(e.g. CCC)}%
}}}
\put(1936,-8161){\makebox(0,0)[lb]{\smash{\SetFigFont{7}{8.4}{\sfdefault}{\mddefault}{\updefault}{\color[rgb]{0,0,0}Consistency checker}%
}}}
\put(846,-7171){\makebox(0,0)[lb]{\smash{\SetFigFont{7}{8.4}{\sfdefault}{\mddefault}{\updefault}{\color[rgb]{0,0,0}(e.g. HOL-CASL)}%
}}}
\put(936,-6766){\makebox(0,0)[lb]{\smash{\SetFigFont{7}{8.4}{\sfdefault}{\mddefault}{\updefault}{\color[rgb]{0,0,0}Theorem prover}%
}}}
\put(8806,-7846){\makebox(0,0)[lb]{\smash{\SetFigFont{7}{8.4}{\sfdefault}{\mddefault}{\updefault}{\color[rgb]{0,0,0}Management of proofs \& change}%
}}}
\put(9091,-7351){\makebox(0,0)[lb]{\smash{\SetFigFont{7}{8.4}{\sfdefault}{\mddefault}{\updefault}{\color[rgb]{0,0,0}Heterogeneous proof engine}%
}}}
\put(10206,-6946){\makebox(0,0)[lb]{\smash{\SetFigFont{7}{8.4}{\rmdefault}{\bfdefault}{\updefault}{\color[rgb]{0,0,0}MAYA}%
}}}
\put(3501,-7216){\makebox(0,0)[lb]{\smash{\SetFigFont{7}{8.4}{\sfdefault}{\mddefault}{\updefault}{\color[rgb]{0,0,0}(e.g. ELAN-CASL)}%
}}}
\put(4051,-6766){\makebox(0,0)[lb]{\smash{\SetFigFont{7}{8.4}{\sfdefault}{\mddefault}{\updefault}{\color[rgb]{0,0,0}Rewriter}%
}}}
\put(5106,-1231){\makebox(0,0)[lb]{\smash{\SetFigFont{7}{8.4}{\rmdefault}{\bfdefault}{\updefault}{\color[rgb]{0,0,0}proposed extensions}%
}}}
\put(5280,-826){\makebox(0,0)[lb]{\smash{\SetFigFont{7}{8.4}{\rmdefault}{\bfdefault}{\updefault}{\color[rgb]{0,0,0}sublanguages and}%
}}}
\end{picture}

\caption{Architecture of the Heterogeneous Tool Set \label{fig:hets}}
\end{center}
\end{figure} 
As shown in Figure \ref{fig:hets} processing structured and
architectural specifications in \Hets\ is independent from processing
the basic (in-the-small) specifications. In order to integrate new
logic one has to provide a parser, static checker and a theorem prover
for the desired logic. Thanks to orthogonal design of \CASL, the
machinery for structured and architectural specifications is generic
and can be used with all the implemented logics.



% ====================================================================
% STATIC ANALYSIS

\chapter{Static Analysis}\label{staticAnalysis}

\section{Institution Independent Semantics}

\CASL\ is at the heart of a family of languages. Sublanguages might be
obtained by imposing syntactic or semantic restrictions on the
original language; the language might as well be extended to support
various paradigms or applications. \CASL\ design follows the
principle of orthogonality of subsequent layers (basic,
structured and architectural specifications and specification
libraries), hence the sublanguages and superlanguages might be produced
by restricting/extending the basic specification language only, while
the other layers remain unchanged. The \nterm{\CASL\ institution}
defined by the semantics of basic specifications can be replaced by another
institution for use with the higher-level specifications, particularly
architectural specifications. This property is reflected by institution
independent semantics style.

\subsection{Institutions and Amalgamation}\label{institutions}

The formalisation of \CASL\ specification semantics
relies on the notion of {\sl institution} \cite{gb1}. An institution
$I$ is a tuple \[(\Sign, \Mod, \Sen, \models)\] where $\Sign$ is a
category of {\sl signatures}, $\Mod:\Sign^{op}\rightarrow\cat{CAT}$ is
a {\sl model functor}, $\Sen:\Sign\rightarrow\cat{Set}$ is a functor
that for a signature $\Sigma$ gives a set of sentences over $\Sigma$
and $\models$ is a family $\{\models_\Sigma\}_{\Sigma\in|\Sign|}$ of
relations on $\Mod(\Sigma)\times\Sen(\Sigma)$, subject to so-called
\nterm{satisfaction condition} (see \cite{gb1}). $\Mod(\Sigma)$ is
referred to as the category of \nterm{$\Sigma$-models} and for a
signature morphism $\sigma:\Sigma_1\rightarrow\Sigma_2$,
$\Mod(\sigma):\Mod(\Sigma_1)\rightarrow\Mod(\Sigma_2)$ is referred to
as a \nterm{reduct functor}. $\Mod(\sigma)(M)$ is often written as
$M|_\sigma$. 

If $\Mod$ maps a cocone in $\Sign$ to a limit in $\cat{CAT}$, the
cocone is said to be \nterm{amalgamable}. If $\Mod$ preserves 
(finite) limits, i.e. (finite) colimit cocones are amalgamable, the 
institution $I$ is said to have the \nterm{(finite) amalgamation
  property}. The $\mathit{SubPCFOL}$ institution (later referred to as
`the \CASL\ institution') that is the formalisation of the \CASL\
underlying logic does not have the finite amalgamation property (see
example in Sect. \ref{staticAnalysis:cellcalc}). 


While the semantics of basic specifications is inherently
institution-specific, the semantics of structured and architectural
specifications is 
independent from the underlying institution. In the following sections
the institution will be referred to implicitly when
ensuring of amalgamability is required.


\subsection{Simple and Extended Static Semantics}\label{semantics_example}

The \CASL\ semantics presented in \cite{refman} is given in two
steps. First, {\sl static semantics} produces syntactic objects for
well-formed phrases; then {\sl model semantics} is applied to a
phrase yielding a class of models of a particular language
construct. While model semantics gives the ultimate meaning of
phrases, it's the static semantics that forms the basis for 
automated static analysis. 

For example, static and model semantics for unit translation
\begin{center}
\texttt{UNIT-TRANSLATION ::= unit-translation UNIT-TERM RENAMING}\\
\end{center}
is defined as follows\footnote{The semantics presented serves the
  purpose of demonstration only --- it will later be compared to the
  extended semantics --- so the exact understanding of all the symbols
  used is not required. For their definitions please refer to
  \cite{refman}.}: 
\paragraph{Static semantics}
\[
\inference{
  \Gamma_s,C_s\vdash\texttt{UNIT-TERM}\triangleright\Sigma\\
  \Sigma\vdash\texttt{RENAMING}\triangleright\sigma:\Sigma\rightarrow\Sigma'
}{
  \Gamma_s,C_s\vdash\texttt{unit-translation UNIT-TERM RENAMING}\triangleright\Sigma'
}
\]
\paragraph{Model semantics}
\[
\inference{
  \Gamma_s,C_s\vdash\texttt{UNIT-TERM}\triangleright\Sigma\\
  \Gamma_s,\Gamma_m,C_s,C\vdash\texttt{UNIT-TERM}\Rightarrow MEv\\
  \Sigma\vdash\texttt{RENAMING}\triangleright\sigma:\Sigma\rightarrow\Sigma'\\
  \textrm{for all } E\in C\textrm{, there exists a unique } M'\in
  \Mod(\Sigma') \textrm{ with } M'|_\sigma = MEv(E)
}{
  \Gamma_s,\Gamma_m,C_s,C\vdash\texttt{unit-translation UNIT-TERM
    RENAMING}\Rightarrow \mathit{Mapping}}
\]
where $\mathit{Mapping} = \{E\mapsto M' | E\in C, M'\in
\Mod(\Sigma'),M'|_\sigma = MEv(E)\}$

It should be noted that the static semantics given above does not discharge
some conditions one would expect to be able to check statically. For
instance the requirement that unique model exists boils down to
ensuring model amalgamability, which is a static condition\footnote{If
  defined as in Sect. \ref{institutions}.}, since the
classes of models considered are restricted only by signatures and
signature morphisms. However, information about the dependencies
between units needs to be preserved in order to perform a more in-depth
analysis.  

Sharing information is represented as a signature diagram --- a
dag (directed acyclic graph) whose nodes are labelled with signatures
and edges are labelled with signature morphisms indicating how the
source unit components are incorporated into the target unit. For a
sample architectural specification 
\begin{EXAMPLE}
\I\ARCHSPEC \NAMEDEFN{Arch\_Spec\_1} =
\begin{ITEMS}
\I\UNITS \[
  U : \NAMEDEFN{Spec\_1};\\
  F : \NAMEDEFN{Spec\_1} \TOTAL \NAMEDEFN{Spec\_2};\\
  G : \NAMEDEFN{Spec\_1} \TOTAL \NAMEDEFN{Spec\_3};
\]
\I\RESULT F[U] \AND G[U]
\end{ITEMS}
\I\END 
\end{EXAMPLE} 
the signature diagram is shown in Figure \ref{fig:diag1}. Please note
that while actual diagram contains signatures and signature morphisms,
the diagram in the figure is labelled with unit terms to indicate
where the signatures and morphisms come from.
\begin{figure}[h]
\vspace{1em}
$$\bfig
\square(-600,
0)|alrb|<600, 600>[\NAMEDEFN{Spec\_1}`U`\NAMEDEFN{Spec\_2}`F[U{]};`F``]
\square(0,
600)|alrb|<600, 600>[\NAMEDEFN{Spec\_1}`\NAMEDEFN{Spec\_3}`U`G[U{]};G```]
\square|alrb|<600, 600>[U`G[U{]}`F[U{]}`F[U{]}\,\AND\,G[U{]};```]
\efig$$
\caption{The signature diagram for \textsc{Arch\_Spec\_1} \label{fig:diag1}}
\end{figure} 
F and G are represented as morphisms from \NAMEDEFN{Spec\_1} to
\NAMEDEFN{Spec\_2} and from \NAMEDEFN{Spec\_1} to \NAMEDEFN{Spec\_3}
respectively. They are instantiatied with U producing F[U] and
G[U]. Finally the amalgamation F[U] \AND G[U] is represented by the
bottom-right node. 

The \nterm{extended static semantics} uses diagram nodes instead of
plain signatures and uses the information about dependencies between
units stored in the diagram to check amalgamability conditions (see
Sect. \ref{sinks} for a formal definition of `ensuring
  amalgamability'). Interpreting diagram $D$ as a functor from its 
\nterm{shape category} $\cat{I}$ to the category $\Sign$ of signatures
the extended static semantics and corresponding model semantics for
unit translation can be written as
\paragraph{Static semantics (extended)}
\[
\inference{
  \Gamma_s,C_s\vdash\texttt{UNIT-TERM}\triangleright(p, D)\\
  D(p)\vdash\texttt{RENAMING}\triangleright\sigma:D(p)\rightarrow\Sigma'\\
  D \textrm{ ensures amalgamability along }
  (\Sigma',\langle\sigma:D(p)\rightarrow\Sigma'\rangle)\\
  D' \textrm{ extends } D \textrm{ by new node } q \textrm{ and edge }
  e:p\rightarrow q \textrm{ with } D'(e) = \sigma
}{
  \Gamma_s,C_s\vdash\texttt{unit-translation UNIT-TERM
    RENAMING}\triangleright(q, D')
}
\]
\paragraph{Model semantics}
\[
\inference{
  \Gamma_s,C_s\vdash\texttt{UNIT-TERM}\triangleright\Sigma\\
  \Gamma_s,\Gamma_m,C_s,C\vdash\texttt{UNIT-TERM}\Rightarrow MEv\\
  \Sigma\vdash\texttt{RENAMING}\triangleright\sigma:\Sigma\rightarrow\Sigma'
  %\strikethrough{\textrm{for all } E\in C\textrm{, there exists a unique } M'\in
  %\Mod(\Sigma') \textrm{ with } M'|_\sigma = MEv(E)}
}{
  \Gamma_s,\Gamma_m,C_s,C\vdash\texttt{unit-translation UNIT-TERM
    RENAMING}\Rightarrow \mathit{Mapping}}
\]
The extended static semantics is much more useful as far as
tools and applications are concerned, as it is able to check a bigger
class of specifications without resorting to theorem proving.


\section{Amalgamability Analysis for \CASL}\label{staticAnalysis:amalg}

The example presented in the previous section shows that the extended
static semantics, being insititution-independent, simply states the
requirement that the diagram must ensure amalgamability. This is a
property dependent on the particular institution and appropriate machinery
discharging this condition has to be provided for the logic used.

In this section the case of \CASL\ institution will be analysed and
a method of checking whether amalgamability is ensured for
given diagram and a set of morphisms extending it will be
presented. The notation introduced in the following paragraphs will
also be used in Chapter \ref{impl} in the description of the implemented
algorithms. 


\subsection{Signature Diagrams and Sinks}\label{sinks}

Let's denote the set of all the nodes in signature diagram $D$ by
$\NodesD$ and the set of all the edges in $D$ by $\EdgesD$. We
say that a family of models $\langle M_p\rangle_{p\in \NodesD}$ is
\nterm{consistent with $D$} if
\begin{itemize}
\item for each $p\in \NodesD$ $M_p\in\Mod(D(p))$ and
\item for each $e:p\rightarrow q\in \EdgesD$ we have $M_p=M_q|_{D(e)}$  ---
  where $D(e)$ is the signature morphism with which 
  edge $e$ is labelled.
\end{itemize}
Consider diagram $D$ where some nodes $p_1,p_2,...,p_k$ labelled
with signatures $\Sigma_1,\Sigma_2,...,\Sigma_k$ ($D(p_i) = \Sigma_i$
for $i\in\{1,2,...,k\}$) are to be amalgamated. Let $\Delta$ stand for
the signature of the amalgamation and
$\tau_i:\Sigma_i\rightarrow\Delta$ denote respective signature
morphisms. A \nterm{sink} $\tau$ is a
pair consisting of the target signature $\Delta$ together with this
family of morphisms: 
$(\Delta, \{\tau_1,\tau_2,...,\tau_k\})$. We say that \nterm{$D$ ensures
amalgamability for $\tau$} if for every family of models $\langle
M_q\rangle_{q\in \NodesD$ consistent with $D$ there exists a unique
  model $M\in\Mod(\Delta)$ such that for all $i\in\{1,2,...,k\}$
  $M|_{\tau_i}=M_{p_i}$. 

The analysis of this condition is discussed in Sect.
\ref{sharing} -- \ref{checking_cell} and the implementation of
algorithms that realise this analysis are presented in Chapter
\ref{impl}. The notation introduced above is used throughout the rest
of this thesis.


\subsection{Sharing Analysis}\label{sharing}

One obvious condition that must be met for the amalgamation to exist
is that all the symbols (of sorts, operations or predicates) of the
component units that are identified in 
the the signature of the amalgamation must share the same
interpretation. This condition is easily decidable through the
analysis of the signature diagram.  

Let $Symbs(D)$ denote the disjoint union 
of all the symbols\footnote{In case of operation and
  predicate symbols we consider the names together with their
  profiles; e.g. two operations named $f$ but having different profiles are
  regarded as different symbols.} 
from $\Sigma_1,\Sigma_2,...,\Sigma_k$, that is a
set of pairs $(p_i, s)$ where $s$ is one of the symbols of
$\Sigma_i$. Let \nterm{morphism path} $\morphpath{\sigma}$ be a
sequence of signature morphisms $\sigma_1,\sigma_2,...,\sigma_m$
determined by a path $\morphpath{e}=e_1,e_2,...,e_m$ in $D$ where
$\sigma_i = 
D(e_i)$. Note that $cod(\sigma_i)=dom(\sigma_{i+1})$ for
$i\in\{1,2...,m-1\}$. We write $dom\morphpath{\sigma}$ to denote the
domain of the first morphism in the sequence and
$cod\morphpath{\sigma}$ to denote the codomain of the last morphism in
$\morphpath{\sigma}$. We also say that $\morphpath{\sigma}$ maps
a symbol $s$ from $dom\morphpath{\sigma}$ to symbol $t$ from
$cod\morphpath{\sigma}$ if the composition of morphisms in
$\morphpath{\sigma}$ maps $s$ to $t$. 
Let's define two equivalence relations on
$Symbs(D)$, $\simeq$ and $\simeq_\tau$, in the following way: 
\begin{itemize}
\item $\simeq$ is the least equivalence satisfying the following
  condition:
  $(p, s)\simeq (q, t)$ if there exists a node $r$ and two 
  paths in the diagram: $\morphpath{e}$ from $r$ to $p$ and
  $\morphpath{f}$ from $r$ to $q$ such that signature
  morphism path $\morphpath{\sigma}$ determined by $\morphpath{e}$
  maps symbol $u$ from $D(r)$ to $s$ from $D(p)$ and morphism path
  $\morphpath{\mu}$ determined by $\morphpath{f}$ maps the same $u$ to
  $t$ from $D(q)$. 
  Informally, this rule states that $s$ and $t$ have a common origin,
  thus denote essentially the same thing. The rule obviously is
  reflexive and symmetrical, so $\simeq$ is just the transitive closure of
  the relation given by this rule.
\item $(p_i, s)\simeq_\tau (p_j, t)$ iff $\tau_i$ maps $s$ to a symbol
  $u$ in $\Delta$ and $\tau_j$ maps $t$ to the same $u$. Informally,
  $(p_i, s)\simeq_\tau (p_j, t)$ means that $s$ and $t$ are identified in the
  amalgamation. 
\end{itemize}
We may now formulate a condition regarding sharing, which is a
necessary prerequisite for amalgamation:
\begin{stdth}\label{th:sharing}
If the condition
\[(*) \simeq_\tau\,\subseteq\,\simeq\]
holds then all the symbols that are identified in the amalgamation have common
interpretations in models from model families consistent with $D$.
\end{stdth}
\proof
Assume that the sharing condition $(*)$ holds. From the definition of
$\simeq_\tau$ it follows that for each pair of identified diagram symbols
$s,t$ from nodes $p_i,p_j$ we have $(p_i, s)\simeq_\tau (p_j,
t)$. Because of $(*)$ we also have $(p_i, s)\simeq (p_j, t)$. 
There are two possible ways in which these diagram symbols got to be
in the $\simeq$ relation: 
\begin{enumerate}
\item either $s$ and $t$ have common origin --- in this case the
  theorem holds trivially, since the origin determines the
  interpretation of both $s$ and $t$,
\item or $s$ and $t$ are equivalent w.r.t. $\simeq$ due to
  transitive closure of the rule that defines $\simeq$ --- this is
  illustrated in Figure \ref{fig:transSimeq}.
\begin{figure}[h]
\vspace{1em}
$$\bfig
\[
\xymatrix{ 
  p_i & & q_1 & & q_2 & ... & q_{m-1} & & q_m & & p_j \\
  & \bullet \ar@{~>}[lu] \ar@{~>}[ru] & & \bullet \ar@{~>}[lu]
  \ar@{~>}[ru] & & & & \bullet \ar@{~>}[lu] \ar@{~>}[ru] & & \bullet
  \ar@{~>}[lu] \ar@{~>}[ru] &  
}
\]
\efig$$
\caption{Transitive closure case in Theorem
  \ref{th:sharing}\label{fig:transSimeq}} 
\end{figure} 
Here symbol $s$ from $D(p_i)$ has common origin with some symbol $u_1$
from $D(q_1)$, $u_1$ has common origin with some $u_2$ from
$D(q_2)$ etc.; finally $u_m$ from $D(q_m)$ and $t$ have
common origin. It follows by induction that $s$, all the $u_i$'s 
and finally $t$ must share common interpretation. 
\end{enumerate}
\QED

\subsection{Cell Calculus}\label{staticAnalysis:cellcalc}

With \CASL\ institution the aspect that turns out to be most complex
and ultimately makes the amalgamability analysis undecidable
is the presence of sort embeddings. Consider following architectural
specification \cite{khtsm1}: 
\begin{EXAMPLE}
\I\SPEC \NAMEDEFN{Num\_1} = \SORTS $Number$, $List[Digit]$, $List[Number]$
\I\SPEC \NAMEDEFN{Num\_2} = \SORTS $Number < List[Digit]$,
$List[Number]$
\I\SPEC \NAMEDEFN{Num\_3} = \SORTS $Number < List[Number]$,
$List[Digit] < List[Number]$
\\
\I\ARCHSPEC \NAMEDEFN{Arch\_Spec\_1} =
\begin{ITEMS}
\I\UNITS \[
  U : \NAMEDEFN{Num\_1};\\
  F : \NAMEDEFN{Num\_1} \TOTAL \NAMEDEFN{Num\_2};\\
  G : \NAMEDEFN{Num\_1} \TOTAL \NAMEDEFN{Num\_3};
\]
\I\RESULT F[U] \AND G[U]
\end{ITEMS}
\I\END 
\end{EXAMPLE} 
\begin{figure}[h]
\vspace{1em}
$$\bfig
\morphism(-1000,600)//[F[U{]}`;]
\Vtriangle(-1400,0)|alr|/`>`/<350,
350>[Number`List[Number{]}`List[Digit{]};\gamma`\alpha`\beta]
\morphism(-250,-100)/-/<0, 800>[`;]
\morphism(350,600)//[G[U{]}`;]
\Vtriangle(0,0)|alr|/>``<-/<350,
350>[Number`List[Number{]}`List[Digit{]};\gamma`\alpha`\beta]
\morphism(1150,-100)/-/<0, 800>[`;]
\morphism(1750,600)//[F[U{]}\,\AND\,G[U{]}`;]
\Vtriangle(1400,0)|alr|/>`>`<-/<350,
350>[Number`List[Number{]}`List[Digit{]};\gamma`\alpha`\beta]
\morphism(-1600, 500)/-/<4000,0>[`;]
\efig$$
\caption{The sort embeddings in F[U], G[U] and F[U] \AND G[U] \label{fig:embs1}}
\end{figure} 
Here the sharing condition for sorts is satisfied, since all the sorts
come from the unit U. However, sort embeddings $\alpha$ in F[U] and
$\beta$, $\gamma$ in G[U] are defined independently, so there is no
guarantee that in the amalgamation the embeddings $\beta\circ\alpha$
and $\gamma$ will commute (see Figure \ref{fig:embs1}). Indeed, under expected
implementation\footnote{$n : Number$ is treated as a list of digits and
  as a one-element long list of numbers,
  $ld : List[Digit]$ is regarded as a list of numbers of the same
  length as $ld$ where each number is one-digit long.} these
injections do not commute. This relates to the lack of amalgamation
property in the \CASL\ institution: the pushout of the signature
diagram (i.e. the union of
F[U] and G[U] signatures) is not mapped by $\Mod$ to a pullback in
$\cat{CAT}$. 

It seems that similar sharing condition to that specified above
for signature objects has to be checked: it must be ensured that the
amalgamation does not assume equality between more embeddings than 
can be deduced from the diagram. However, as the example above shows,
not only individual embeddings declared explicitly would have to be
analysed, but also their compositions (\nterm{embedding paths}). That
leads to undecidability, since there might exist embedding loops that
generate infinite number of possible paths\footnote{Infinite number of
paths doesn't necessarily mean the problem presented is
undecidable. However, the rules of cell calculus directly correspond
to the axioms of monoids with left cancellation and the (undecidable) problem of
deciding whether such a monoid defined by a finite set of equations is
trivial can be reduced to the cell condition presented below (see \cite{khtsm1}).}.

Finiteness aside, the problem might be formalised in a fashion similar
to the sharing condition, bearing in mind that the rules according to
which two 
embedding paths are considered to be equivalent are somewhat more
complex than in case of $\simeq$ and $\simeq_\tau$ relations. 
An \nterm{embedding path} in diagram $D$ is a sequence of
embeddings $\alpha_1, \alpha_2,...\alpha_m$ such that the domains and
codomains of consecutive embeddings are equivalent w.r.t. supplied
$\simeq$ relation (on the disjoint union of sort symbols from $D$). We
define a \nterm{cell} to be a pair of embedding paths with equivalent
beginning sorts and equivalent ending sorts; the
paths in the cell are considered to be equivalent. We will need
to show that the set of cells generated by the diagram is a superset
of the set of cells induced by the amalgamation. 

The considerations that follow assume that the only symbols in
the signatures are sort symbols; operations and predicates are
irrelevant for sort embedding analysis and would only clutter up the
notation. 
First, consider the simple extension of $\simeq$ to all the sorts in
the diagram: the least equivalence on the disjoint union of the sort
symbols in $D$ such that for all
diagram edges $e : p\rightarrow q$ and sort $s$ in $D(p)$, $(p,
s)\simeq(q, D(e)(s))$. Let $\msym{Embs}(D)$ be the disjoint union of all the sort
embeddings defined in the signatures in $D$, i.e. a set of pairs $(p,
\alpha)$ where $\alpha$ is an embedding in $D(p)$. The embedding 
paths mentioned above are formally defined as finite words over
$\msym{Embs}(D)$ in form $\seq{(p_n, \alpha_n),...,(p_1, \alpha_1)}$ such that for
$1\leq i < n$ $(p_i, cod(\alpha_i))\simeq(p_{i+1},
dom(\alpha_{i+1}))$. Let's denote the set of all such
words by $Adm_\simeq$\footnote{$Adm_\simeq$ stands for
  \nterm{$\simeq$-admissible}; the words are constructed based on the
  $\simeq$ relation.}. Note that the first morphism in a path (in the
diagrammatic order) is the rightmost letter in the word. For a 
word $\omega = \seq{(p_n, \alpha_n),...,(p_1, \alpha_1)}$ we write
$dom(\omega)$ for $(p_1, dom(\alpha_1))$ and $cod(\omega)$ for $(p_n,
cod(\alpha_n))$. Now, a \nterm{cell over $D$} may be defined as a pair
$(\omega,\upsilon)$ of words from $Adm_\simeq$ such that $dom(\omega)\simeq
dom(\upsilon)$ and $cod(\omega)\simeq cod(\upsilon)$.

Let $\cong$ be the least relation on $Adm_\simeq$ satisfying
following rules:
\begin{eqnarray}
\nonumber
\inference[\rulename{Refl}]{\omega\in Adm_\simeq}{\omega\cong\omega}
&
\inference[\rulename{Symm}]{\omega\cong\upsilon}{\upsilon\cong\omega}
&
\inference[\rulename{Trans}]{\omega\cong\upsilon &
  \upsilon\cong\psi}{\omega\cong\psi}
\end{eqnarray}
\begin{eqnarray}
\nonumber
\inference[\rulename{Diag}]{l : p\rightarrow q\,\text{is an edge in}\,
  D}{\seq{(p, e)}\cong\seq{(q, D(l)(e))}}
&
\inference[\rulename{Comp}]{ed\,\text{is defined in}\,D(p)}{\seq{(p,
    e), (p, d)}\cong\seq{(p, ed)}} 
\end{eqnarray}
\begin{eqnarray}
\nonumber
\inference[\rulename{Cong}]{\omega\cong\upsilon & \psi\cong\phi &
  cod(\omega)\simeq dom(\psi)}{\psi\omega\cong\phi\upsilon}
&
\inference[\rulename{Lc}]{\psi\omega\cong\psi\upsilon}{\omega\cong\upsilon}
\end{eqnarray}
Obviously (due to \rulename{Refl}, \rulename{Symm} and
\rulename{Trans} rules) $\cong$ is an equivalence relation.
$\cong$ can be thought of as a set of cells over $D$, the seven rules used
to inductively define this set are therefore called the \nterm{cell
  calculus}. 

In this way we have defined the relation that represents
the expected equivalences between embedding paths. What still needs to be
done is to represent the paths that are considered equivalent in the
amalgamation, that is paths that have equivalent beginnings and
ends in the amalgamation. Let $Adm_{\simeq_\tau}$ be the set of all
the words defined in the same manner as $Adm_\simeq$ but where the
domains and codomains of consecutive morphisms are equivalent
w.r.t. $\simeq_\tau$ (and so $Adm_{\simeq_\tau}$ are words of
embeddings in signatures $\Sigma_1,\Sigma_2,...,\Sigma_k$). We may now define
$\cong_\tau$ to be the least 
equivalence on $Adm_{\simeq_\tau}$ where for $\omega,\upsilon\in
Adm_{\simeq_\tau}$ if $dom(\omega)\simeq_\tau dom(\upsilon)$ and
$cod(\omega)\simeq_\tau cod(\upsilon)$ then
$\omega\cong_\tau\upsilon$. The relation $\cong_\tau$ represents
embedding paths that are identified in the amalgamation. Note that if
the sharing condition $(*)$ holds then $Adm_{\simeq_\tau}\subseteq
Adm_\simeq$. 
\begin{stdth}\label{th:amalg}
In the above setting, if both the sharing condition $(*)$ and the \nterm{cell condition}:
\[(**) \cong_\tau\,\subseteq\,\cong\] 
hold then $D$ ensures amalgamability for $\tau$.
\end{stdth}
\proof
See the extended version of \cite{khtsm1}.
\QED

\subsection{Optimisations}\label{impl:amalg:opt}

The rules of cell calculus presented in
Sect. \ref{staticAnalysis:cellcalc} capture the formal aspect of
amalgamability analysis; however, using them naively is inefficient as far as
the implementation is concerned. In particular, the rules of extended static
semantics often generate a number of diagram nodes with similar
signatures and, as a consequence, the $\simeq$ relation has large
equivalence classes. This leads to large $Adm_\simeq$ set, since
embeddings with equivalent domains and codomains are equivalent
and all combinations of equivalent embeddings need to be used in
order to obtain the set of all $\simeq$-admissible words. Large
$Adm_\simeq$ coupled with expensive algorithm for computing $\cong$
relation results in long execution time.

A way to remedy this is to observe that \rulename{Diag} rule of cell
calculus corresponds to a trivial equivalence of embeddings. We may
therefore compute an equivalence $\sim$ on $\msym{Embs}(D)$ defined
by this rule and factor $\msym{Embs}(D)$ through $\sim$. Let
$\msym{CanonicalEmbs}(D)$ denote the set $\msym{Embs}(D)/\sim$. For
checking the cell condition it is now sufficient to consider the set
of $\simeq$-admissible words over $\msym{CanonicalEmbs}(D)$. The
argument is that any equivalence proof $\omega\cong\upsilon$ in the
cell calculus can be normalised to a proof
$\omega\cong\phi\cong\psi\cong\upsilon$, where $\phi$ and $\psi$ are
$\simeq$-admissible words over
$\msym{CanonicalEmbs}(D)$. $\phi\cong\psi$ is proved using words over
$\msym{CanonicalEmbs}(D)$ only. However, this proof requires a
slightly different set of rules; instead of \rulename{Comp} and
\rulename{Diag} rules we use a combination of these:
\begin{eqnarray}
\nonumber
\inference[\rulename{CompDiag}]{(p,e)\sim(p,e') & (q,d)\sim(n,d') &
  (m,f)\sim(n,e'd')}{\seq{(p,e), (q,d)}\cong\seq{(m,f)}}
\end{eqnarray}
Now the equivalences $\omega\cong\phi$ and $\psi\cong\upsilon$ are
proved through obvious applications of \rulename{Diag} and \rulename{Cong}
rules only. 
A further refinement is to close the $\sim$ relation w.r.t. congruence
rule for single embeddings, namely:
\begin{eqnarray}
\nonumber
\inference[\rulename{Cong1}]{(p,e)\sim(q,e') & (p,d)\sim(q,d') &
  (p,ed) \text{ and } (q,e'd') \text{ are in the domain of }
  \sim}{(p,ed)\sim(q,e'd')}
\end{eqnarray}
This reduces the set $\msym{CanonicalEmbs}(D)$ even more, while the
argument above remains valid.

The implementation described in Chapter \ref{impl} uses this
optimisation, i.e. $\cong$ is computed over  
$\msym{CanonicalEmbs}(D)$ and $\cong_\tau$ is translated to
$\msym{CanonicalEmbs}(D)$. 


\subsection{Checking the Cell Condition}\label{checking_cell}

The domains of $\cong$ and $\cong_\tau$ may be infinite and
while we could find a finite subrelation $\cong_\tau'$ of $\cong_\tau$
such that if $\cong_\tau'\,\subseteq\,\cong$ then
$\cong_\tau\,\subseteq\,\cong$, 
(it is sufficient to consider loopless words in the domain of
$\cong_\tau$\footnote{Note that the identity embeddings are essentially
  trivial w.r.t. $\cong$ and $\cong_\tau$, we can therefore ommit them in
  further considerations.}) in general it is not possible to finitely represent
$\cong$.

Anyway, attempts have been made in \cite{khtsm1} to distinguish
special cases where the cell condition can be effectively
analysed.
\begin{enumerate}
\item When the set $Adm_\simeq$ is finite (i.e. embeddings do not form
  non-trivial cycles) the relation $\cong$ is finite. Computing $\cong$ is
  PSPACE-hard, however, an observation that if the inclusion
  $\simeq_\tau\,\subseteq\,\simeq$ holds and the colimit of $D$ (as a
  diagram in the category of left cancellable small
  categories\footnote{We take the category $\cat{lcCat}$ of left 
  cancellable small categories to be the category of signatures (we
  interpret preorders as thin categories), hence
  the diagram is a fuctor $D:\cat{I}\rightarrow\cat{lcCat}$.}) is a
  thin category
  than the cell condition is satisfied leads to a polynomial time
  algorithm.  
\item Consider a subrelation $\cong_0$ of $\cong$ that is the least
  equivalence satisfying all the rules of cell calculus except for
  \rulename{Cong} and \rulename{Lc}. Obviously
  $\cong_\tau\,\subseteq\,\cong_0$ implies the cell
  condition. Although $\cong_0$ has possibly infinite domain it has only
  finitely many non-reflexive elements and can therefore be finitely
  represented. 
\item Let $\cong^R$ denote the relation $\cong$ restricted to loopless
  words. The domain of $\cong^R$ is finite, thus the relation itself
  is finite, so it is possible to devise a (PSPACE-hard) algorithm
  that checks the inclusion $\cong_\tau\,\subseteq\,\cong^R$.
\end{enumerate}
These observations form the basis for an algorithm that is able to
verify the cell condition for most of practically useful cases of 
architectural specifications. The least trivial part of the
implementation presented in the next chapter is entirely based on
these results. 



% ====================================================================
% IMPLEMENTATION

\chapter{The Implementation}\label{impl}

\section{Module Overview}

The source code of \Hets\ is split into Haskell modules. The hierarchy
of modules is reflected by the hierarchy of directories in which the
source files are stored, i.e. the module \module{Static.ArchDiagram}
corresponds to the file \path{Static/ArchDiagram.hs} in the \Hets\
source code directory.

The modules that form the foundations of \Hets\ are:
\begin{itemize}
\item \module{Logic.*} --- provide data structures for logics and
  logic comorphisms and define Grothendieck logic --- a heterogeneous
  logic over which the data structures and algorithms for
  specifications in-the-large are built.
\item \module{Comorphisms.*} --- define comorphisms between different
  logics implemented in \Hets\ and assemble all these logics into the
  \nterm{logic graph}.
\item \module{Syntax.*} --- define the abstract syntax and parsers for
  \CASL\ structured and architectural specifications and specification
  libraries.
\item \module{Static.*} --- contain the static analysis algorithms for
  specifictaions in-the-large.
\end{itemize}

In addtition there are logic-specific modules in \module{CASL.*},
\module{HasCASL.*}, \module{CspCASL.*} etc.; a separate directory is
provided for each logic supported by \Hets. A library of common data
structures (maps, graphs etc.) is provided by modules in
\module{Common.Lib.*}. 

The modules implemented as a part of this thesis are:
\begin{itemize}
\item \module{Syntax.Parse\_AS\_Architecture} --- parsing of
  architectural specifications,
\item \module{Static.ArchDiagram} --- definition of signature diagrams,
\item \module{Static.AnalysisArchitecture} --- institution independent
static analysis of architectural specifications,
\item \module{CASL.Amalgamability} --- amalgamability analysis for
  \CASL\ institution.
\end{itemize}


\section{Parsing}

Parsing in \Hets\ is done using \nterm{monadic parser combinator}
\cite{hm1} library Parsec\footnote{See
  \webaddr{http://www.cs.uu.nl/~daan/parsec.html}.}. The abstract
syntax elements for architectural specifications are defined in
\module{Syntax.AS\_Architecture}. Functions building the \nterm{abstract
syntax tree} (AST) are defined in
\module{Syntax.Parse\_AS\_Architecture}. They follow the
abstract syntax defined in \cite{refman}; the grammar has only been
slightly adjusted to eliminate left-hand-side recursion and
disambiguate some constructs. There's one function for each
non-terminal symbol of the grammar.
\begin{verbatim}
-- | Parse group unit term
-- @
-- GROUP-UNIT-TERM ::= UNIT-NAME
--                   | UNIT-NAME FIT-ARG-UNITS
--                   | { UNIT-TERM }
-- @
groupUnitTerm :: (AnyLogic, LogicGraph) -> AParser UNIT_TERM
groupUnitTerm l =
        -- unit name/application
    do name <- simpleId
       (args, pos) <- fitArgUnits l
       return (Syntax.AS_Architecture.Unit_appl name args pos)
    <|> -- unit term in brackets
    do lbr <- asKey "{"
       ut <- unitTerm l
       rbr <- asKey "}"
       return (Syntax.AS_Architecture.Group_unit_term ut 
                                 (map tokPos [lbr, rbr]))
\end{verbatim}
The function presented above parses a group unit term. The
{\tt<|>} symbol is a Parsec alternative that separates
productions. The first two productions are captured in the first part
of the alternative --- \sym{fitArgUnits} may return an empty list
and hence yield unit name. The code indeed looks very similar to the
context-free grammar. 

Special care needs to be taken when parsing unit specifications:
\begin{verbatim}
UNIT-SPEC ::= SPEC-NAME 
            | UNIT-ARGS GROUP-SPEC 
            | arch spec GROUP-ARCH-SPEC 
            | closed UNIT-SPEC
\end{verbatim}
The first two cases are indistinguishable at the parsing stage, since
{\tt UNIT-ARGS} might be empty and {\tt GROUP-SPEC} might reduce
to {\tt SPEC-NAME}. Therefore in either of these cases the
\sym{unitSpec} function parses a unit type 
(i.e. tries the second production). At the static analysis stage the
correct production is regonised depending on the context and
appropriate correction to the AST is made.


\section{Static Analysis}

The (institution independent) static analysis of architectural
specifications is based on the extended static semantics defined in
\cite{refman} (see Sect. \ref{semantics_example}). 

\subsection{Data Structures}
Data structures used in the analysis are defined in
\module{Static.ArchDiagram}. The \nterm{development graphs} defined in
\module{Static.DevGraph} are used as well. Both development graphs and
signature diagrams are represented using \sym{Graph} data
type defined in \module{Common.Lib.Graph}.

The \sym{DGraph} type defined in \module{Static.DevGraph} is a graph whose
nodes are labelled with signatures and edges are labelled with
signature morphisms. It represents the dependencies between structured
specifications: the leaves represent basic specifications and the inner
nodes are structured specifications constructed from other
specifications. The development graph also contains \nterm{theorem
  links} --- edges to which proof obligations are attached.
The \sym{NodeSig} type is a representation of signature that takes into
account the development graph. \sym{NodeSig} contains both a node in
development graph and a signature with which that node is labelled.
The operations on \sym{DGraph} defined in \module{Static.DevGraph} are:
\begin{itemize}
\item \sym{extendDGraph} --- extends given \sym{DGraph} with given morphism
  originating from given node.
\item \sym{extendDGraphRev} --- extends given \sym{DGraph} with given morphism
  whose codomain is in given node.
\item \sym{nodeSigUnion} --- extends given \sym{DGraph} with a node
  representing union of signatures from given set of nodes.
\end{itemize}

The \sym{Diag} type defined in \module{Static.ArchDiagram} is a graph
whose nodes are labelled with \sym{NodeSigs} and edges are labelled with
signature morphisms. The signature stored in a node is the signature
of the unit that node represents.
The operations on \sym{Diag} defined in \module{Static.ArchDiagram} are:
\begin{itemize}
\item \sym{extendDiagram} --- extends given \sym{Diag} with a node
  containing given \sym{NodeSig} and an edge from given node to the
  new node. The edge is labelled with supplied signature morphism.
\item \sym{extendDiagramIncl} --- extends given \sym{Diag} with a node containing
  given \sym{NodeSig} and a set of edges from given set of nodes to the
  new node. The edges are labelled with signature inclusions. 
\item \sym{extendDiagramWithMorphism} --- extends given \sym{Diag} with an
  edge from given node and a new node that is the target of the edge
  added. The edge is labelled with given signature morphism and the
  new node contains the codomain of this morphism.
\item \sym{extendDiagramWithMorphismRev} --- similar to
  \sym{extendDiagramWithMorphism}, but the new node is at the begginning
  of the new edge and contains the domain of the morphism.
\item \sym{homogeniseDiagram} --- tries to coerce the logics of all the
  signatures and morphisms in given \sym{Diag} to given logic using
  coertion function \sym{rcoerce} from \module{Logic.Logic}.
\item \sym{homogeniseSink} --- tries to coerce the logics of all the
  morphisms in given list of signature morphisms to given
  logic using \sym{rcoerce} function from \module{Logic.Logic}. 
\end{itemize}


\subsection{Functions}
Functions that perform the analysis are defined in 
\module{Static.AnalysisArchitecture}. There is essentially one
function for each semantic rule. The only exported functions are
\sym{ana\_ARCH\_SPEC} and \sym{ana\_UNIT\_SPEC}; given logic graph, global
context (containing current development graph), current logic and
appropriate fragment of the abstract syntax tree these functions return
architectural/unit signature, modified development graph and modified
AST fragment. 

The checking of amalgamability condition is realised by
\sym{assertAmalgamability} and \\\sym{homogeneousEnsuresAmalgamability}
functions. Given a diagram and a sink they homogenise the diagram and
the sink by
coercing the logic of each signature and each morphism to the logic of
the target of the 
sink and then passing it to the \sym{ensures\_amalgamability} function for
this logic. \sym{ensures\_amalgamability} is defined as a part of
\sym{StaticAnalysis} type class, to which all the logics implemented in
\Hets\ belong.


\section{Amalgamability Analysis for \CASL}
The amalgamability checking for \CASL\ (specific for the \CASL\
institution) is based on the algorithm
presented in \cite{khtsm1}. All the code responsible for amalgamability
analysis is in \module{CASL.Amalgamability}.

\subsection{Data Structures}\label{impl:amalg:ds}

The data structures used extensively throughout the amalgamabilty
checking algorithm are equivalence relations. Two types corresponding
to two different representations of these have been defined.
\begin{itemize}
\item \sym{EquivRel\;a} is a list of equivalence classes
  (\code{[EquivClass\;a]}). \sym{EquivClass\;a} is a list of elements
  of type \sym{a}. This is the representation commonly used in the
  algorithm, particularly when inclusion of one equivalence into
  another is to be checked.
\item \sym{EquivRelTagged\;a\;b} is a list of pairs where the first element
  has type \sym{a} and the second has type \sym{b} (\sym{[(a,
  b)]}). The first element of a pair is 
  interpreted as an element of relation domain, the second is an
  equivalence class tag. Pairs with equal second elements represent
  elements of the same equivalence class. This representation is used
  while computing relations --- it makes iterating through the
  elements of relation domain easy, since merging equivalence
  classes does not change the order of elements in the list.
\end{itemize}
\sym{CASLDiag} is defined as a diagram (graph) whose nodes are
labelled with \CASL\ signatures (\sym{CASLSign}) and edges are
labelled with \CASL\ signature morphisms (\sym{CASLMor}). Types have
been defined as well for elements of disjoint unions of sorts, operations,
predicates and sort embeddings from the diagram; these are
\sym{DiagSort}, \sym{DiagOp}, \sym{DiagPred} and \sym{DiagEmb}
respectively. \sym{DiagEmbWord} is a list of \sym{DiagEmbs}.

\subsection{Auxiliary Functions}\label{impl:amalg:af}

The operation commonly performed throughout the algorithm is computing
equivalence relations. It is usually done according to Algorithm
\ref{alg:equiv}.
\begin{algorithm}[H]
\caption{Computing an equivalence relation\label{alg:equiv}}
\dontprintsemicolon
\AlgData{$\bullet$ a set $E$ of elements\\$\bullet$ a set $R$ of rules}
\AlgResult{the least equivalence on $E$ satisfying all the rules in
  $R$ represented as \sym{EquivRel}}
\BlankLine
\lnl{al:baseEquiv} $\msym{rel}\leftarrow$ the least equivalence on $E$ represented as
\sym{EquivRelTagged}\;
\ForEach{$\msym{mergeRule}\in R$}{
  \lnl{al:merge} merge the equivalence classes in \sym{rel} according to \sym{mergeRule}\;
}
\lnl{al:conv} convert \sym{rel} from \sym{EquivRelTagged} to \sym{EquivRel}\;
\return{\sym{rel}}\;
\end{algorithm}

Computing the equivalence in line \algline{1} is straightforward: 
based on set $E$ (represented as a list) we need to create a list
containing for each element $e\in E$ a pair $(e,e)$. This list would
represent a relation where each element is in a different equivalence
class, hence is in relation only with itself (see
\ref{impl:amalg:ds} for explanation how an equivalence relation is
represented using \sym{EquivRelTagged}). The conversion in line
\algline{3} is performed by \sym{taggedValsToEquivClasses} function;
the opposite conversion is also possible and is done by
\sym{equivClassesToTaggedVals}. 
Each element of $R$ is a function that given two elements $e_1,e_2\in
E$ returns \sym{True} if $e_1$ and $e_2$ should be in relation
according to the rule it represents and \sym{False}
otherwise. Merging of equivalence classes from line \algline{2} is
performed by \sym{mergeEquivClassesBy} function (described in
Algorithm \ref{alg:mergeEquivClassesBy}). Note, that
\sym{mergeRule} passed to Algorithm \ref{alg:mergeEquivClassesBy} does not
depend on the current state of the relation 
whose closure is being generated, hence it is possible to compute the
closure in one pass, without resorting to fixpoint computation.
\begin{algorithm}[H]
\caption{\sym{mergeEquivClassesBy} function\label{alg:mergeEquivClassesBy}}
\dontprintsemicolon
\AlgData{$\bullet$ a function $\msym{mergeRule} : E \rightarrow E
  \rightarrow Bool$\\$\bullet$ a relation \sym{rel} represented as
  \sym{EquivRelTagged\;E\;T}} 
\AlgResult{the least equivalence \sym{rel'} satisfying \sym{mergeRule}
  such that $\msym{rel}\subseteq \msym{rel'}$}
\BlankLine
\ForEach{$(e_1, t_1)\in \msym{rel}$}{
  \ForEach{$(e_2, t_2)\in \msym{rel}$}{
    \If{$\msym{mergeRule}\;e_1\;e_2$}{
      $\msym{rel}\leftarrow \msym{mergeEquivClasses}\;\msym{rel}\;t_1\;t_2$\;
    }
  }
}
\return{\sym{rel}}\;
\end{algorithm}
\begin{algorithm}[H]
\caption{\sym{mergeEquivClasses} function\label{alg:mergeEquivClasses}}

\dontprintsemicolon
\AlgData{$\bullet$ a relation \sym{rel} represented as
  \sym{EquivRelTagged\;E\;T}\\$\bullet$ a tag $t_1 : T$\\$\bullet$ a tag
  $t_2 : T$} 
\AlgResult{the relation \sym{rel} with equivalence classes represented by
  $t_1$ and $t_2$ merged}
\BlankLine
\ForEach{$(e, t)\in \msym{rel}$}{
  \lIf{$t = t_2$}{
    replace $(e, t)$ by $(e, t_1)$ in \sym{rel}\;
  }
}
\return{\sym{rel}}\;
\end{algorithm}

Another operation performed often is checking whether an inclusion
between two relations holds. This is done by \sym{subRelation}
function described in Algorithm \ref{alg:subRelation}.
\begin{algorithm}[H]
\caption{\sym{subRelation} function\label{alg:subRelation}}
\dontprintsemicolon
\AlgData{$\bullet$ an equivalence \sym{sub} represented as
  \sym{EquivRel}\\$\bullet$ an equivalence \sym{sup} represented as 
  \sym{EquivRel}} 
\AlgResult{\sym{Just (e_1, e_2)} where $(e_1, e_2)\in \msym{sub}$ and $(e_1,
  e_2)\not\in \msym{sup}$ or \sym{Nothing} if $\msym{sub}\subseteq \msym{sup}$} 
\BlankLine
\ForEach{equivalence class $\msym{eqcl}_\msym{sub}$ from \sym{sub}}{
  $e_1\leftarrow$ the first element of $\msym{eqcl}_\msym{sub}$\;
  \eIf{there exists an equivalence class in \sym{sup} containing
    $e_1$}{
    $\msym{eqcl}_\msym{sup}\leftarrow$ the equivalence class from \sym{sup} containing $e_1$\;
  }{
    $\msym{eqcl}_\msym{sup}\leftarrow$ an empty list\;
  }
  \ForEach{$e_2$ from $\msym{eqcl}_\msym{sub}$}{
    \lIf{$e_2\not\in \msym{eqcl}_\msym{sup}$}{\return{\sym{Just (e_1, e_2)}}\;}
  }
}
\return{\sym{Nothing}}\;
\end{algorithm}


\subsection{Functions}

The main function in the module is \sym{ensuresAmalgamability}. It
performs the amalgamability check based on the algorithm outlined in 
\cite{khtsm1} and is a realisation of the analysis presented in
Sect. \ref{staticAnalysis:amalg}; for the meaning of symbols used in
Algorithm  
\ref{alg:ensuresAmalg} please refer to that section. Note that
$\simeq$ (and, correspondingly,  
$\simeq_\tau$) have been split to three relations: $\simeq$ for sorts,
$\simeq^{Op}$ for operations and $\simeq^{Pred}$ for
predicates. Subsort embeddings are dealt with using $\cong$ and
$\cong_\tau$ relations. 
\begin{algorithm}[H]
\caption{\sym{ensuresAmalgamability} function\label{alg:ensuresAmalg}}
\dontprintsemicolon
\AlgData{$\bullet$ diagram $D$\\$\bullet$ sink $\tau$ --- a list of
  pairs $(p_i, \tau_i)$, where $p_i$ is a node in $D$ and $\tau_i:D(p_i)\rightarrow\Sigma$ is a
  signature morphism; $\Sigma$ is common for all the morphisms in the sink}
\AlgResult{\sym{Yes}, \sym{No} or \sym{DontKnow}, saying whether $D$ ensures amalgamability for $\tau$}
\BlankLine
\nl compute $\simeq$ for $D$ and $\simeq_\tau$ for $D$, $\tau$\;
\lIf{$\simeq_\tau\,\not\subseteq\,\simeq$}{
  \return{\sym{No}}\;
}\Else{
  \nl compute $\simeq^{Op}$ for $D$ and $\simeq^{Op}_\tau$ for $D$, $\tau$\;
  \lIf{$\simeq^{Op}_\tau\,\not\subseteq\,\simeq^{Op}$}{
    \return{\sym{No}}\;
  }\Else{
    \nl compute $\simeq^{Pred}$ for $D$ and $\simeq^{Pred}_\tau$ for $D$, $\tau$\;
    \lIf{$\simeq^{Pred}_\tau\,\not\subseteq\,\simeq^{Pred}$}{
      \return{\sym{No}}\;
    }\Else{
      \nl compute $\cong_\tau$ for $\tau$, $\simeq_\tau$\;
      \nl $\cong^0_\tau\leftarrow\cong_\tau$ restricted to non-reflexive
      elements\;  
      \nl compute $\cong_0$ for $D$, $\simeq$\;
      \lIf{$\cong^0_\tau\,\subseteq\,\cong_0$}{
	\return{\sym{Yes}}\;
      }\Else{
	compute $\msym{Embs}(D)$ for $D$\;
	\nl compute $\sim$ for $D$, $\msym{Embs}(D)$\;
	\nl $\cong^C_\tau\leftarrow\cong_\tau$ translated to canonical
	embeddings using $\sim$\;
	\nl\eIf{$Adm_\simeq$ is finite}{
	  \nl compute $Adm_\simeq$ for $\msym{Embs}(D)$, $\simeq$\;
	  \nl\lIf{colimit of $D$ is thin (for $\simeq$,
	    $\msym{Embs}(D)$, $\cong_0$)}{
	    \return{\sym{Yes}}\;
	  }\Else{
	    \nl compute $\msym{CanonicalEmbs}(D)$ for $\sim$\;
	    \nl compute $Adm^C_\simeq$ for $\msym{CanonicalEmbs}(D)$, $\simeq$\;
	    \nl compute $\cong$ for $Adm^C_\simeq$, $\simeq$, $\sim$\;
	    \lIf{$\cong^C_\tau\,\subseteq\,\cong$}{
	      \return{\sym{Yes}}\;
	    }\lElse{
	      \return{\sym{No}}\;
	    }
	  }
	}{
	  \nl compute $\cong^R$ for $D$, $\simeq$, $\sim$\;
	  \lIf{$\cong^C_\tau\,\subseteq\,\cong^R$}{
	    \return{\sym{Yes}}\;
	  }\lElse{
	    \return{\sym{DontKnow}}\;
	  }
	}
      }
    }
  }
}
\end{algorithm}

First (line \algline{1} of Algorithm \ref{alg:ensuresAmalg}),
equivalences $\simeq$ and $\simeq_\tau$ need to be 
computed. This is performed by \sym{simeq} and \sym{simeq\_tau}
respectively (Algorithms \ref{alg:simeq} and
\ref{alg:simeqtau}). Computations of $\cdot^{Op}$ and $\cdot^{Pred}$
variants (lines 
\algline{2} and \algline{3} in Algorithm \ref{alg:ensuresAmalg}) of
$\simeq$ and $\simeq_\tau$ are done in exactly the same way.
\begin{algorithm}[H]
\caption{\sym{simeq} function\label{alg:simeq}}
\dontprintsemicolon
\AlgData{diagram $D$}
\AlgResult{the $\simeq$ relation represented as \sym{EquivRel\;
    DiagSort}}
\BlankLine
$\msym{Sorts}(D)\leftarrow$ the set of all the sorts in $D$\;
$rel\leftarrow$ the least equivalence on $\msym{Sorts}(D)$ represented as
\sym{EquivRelTagged}\;
$\msym{mergeCond}\leftarrow\lambda s.\lambda s'.$ ``there exists morphism in
$D$ that maps $s$ to $s'$ or $s'$ to $s$''\;
$\msym{rel}\leftarrow \msym{mergeEquivClassesBy}\;\msym{mergeCond}\;\msym{rel}$\;
$\msym{rel}\leftarrow \msym{taggedValsToEquivClasses}\;\msym{rel}$\;
\return{\sym{rel}}\;
\end{algorithm}
\begin{algorithm}[H]
\caption{\sym{simeq\_tau} function\label{alg:simeqtau}}
\dontprintsemicolon
\AlgData{$\bullet$ the diagram $D$\\$\bullet$ the sink $\tau$ --- a list of
  pairs $(p_i, \tau_i)$, where $p_i$ is a node in $D$ and $\tau_i:D(p_i)\rightarrow\Sigma$ is a
  signature morphism; $\Sigma$ is common for all the morphisms in the sink}
\AlgResult{the $\simeq_\tau$ relation represented as \sym{EquivRel\;
    DiagSort}}
\BlankLine
$\msym{rel}\leftarrow$ an empty list\;
\ForEach{$(p_i, \tau_i)\in\tau$}{
  \ForEach{sort $s$ from $D(p_i)$}{
    $t\leftarrow \tau_i(s)$\;
    add a pair $((p_i,s),t))$ to \sym{rel}\;
  }
}
$\msym{rel}\leftarrow \msym{taggedValsToEquivClasses}\;rel$\;
\return{\sym{rel}}\;
\end{algorithm}

The next point after checking the sharing condition is computing
$\cong_\tau$, $\cong^0_\tau$ and $\cong_0$ (lines
\algline{4}-\algline{6}). $\cong_\tau$ is computed by \sym{cong\_tau}
function (Algorithm \ref{alg:congtau}).
\begin{algorithm}[H]
\caption{\sym{cong\_tau} function\label{alg:congtau}}
\dontprintsemicolon
\AlgData{$\bullet$ the diagram $D$\\$\bullet$ the sink $\tau$ --- a
  list of pairs $(p_i, \tau_i:D(p_i)\rightarrow\sigma)$\\$\bullet$ the
  $\simeq_\tau$ relation represented as \sym{EquivRel\;DiagSort}}
\AlgResult{the $\cong_\tau$ relation represented as \sym{EquivRel\;
    DiagEmbWord}}
\BlankLine
$\msym{Embs}(\tau)\leftarrow$ the set of all pairs $(p, e)$ such that $p$ is
a node in one of the pairs from $\tau$\\
\Indp and $e$ is an embedding in $D(p)$\;\Indm
$W\leftarrow \msym{looplessWords}\;\msym{Embs}(\tau)\;\simeq_\tau$\;
$\msym{rel}\leftarrow$ the least equivalence on $W$\;
$\msym{mergeCond}\leftarrow\lambda\omega.\lambda\upsilon.dom(\omega)\simeq_\tau
dom(\upsilon)$ and $cod(\omega)\simeq_\tau cod(\upsilon)$\;
$\msym{rel}\leftarrow \msym{mergeEquivClassesBy}\;\msym{mergeCond}\;\msym{rel}$\;
$\msym{rel}\leftarrow \msym{taggedValsToEquivClasses}\;\msym{rel}$\;
\return{\sym{rel}}\;
\end{algorithm}
In order to obtain $\cong^0_\tau$ it's sufficient to filter out these
equivalence classes from $\cong_\tau$ that have just one
element. Generating $\cong_0$ boils down to building the least
equivalence on one-letter words (i.e. single embeddings), merging
equivalence classes according to \rulename{Diag} rule and finally
closing the relation w.r.t. \rulename{Comp} rule by generating all the
two-letter words and inserting them into appropriate equivalence
classes. This is illustrated in Algorithm \ref{alg:cong0}. The
\sym{diagRule} in line \algline{1} is a (simple to construct) functional
representation of the \rulename{Diag} rule.
\begin{algorithm}[H]
\caption{\sym{cong\_0} function\label{alg:cong0}}
\dontprintsemicolon
\AlgData{$\bullet$ the diagram $D$\\$\bullet$ the
  $\simeq$ relation represented as
  \sym{EquivRel\;DiagSort}}
\AlgResult{the $\cong_0$ relation represented as \sym{EquivRel\;
    DiagEmbWord}}
\BlankLine
$\msym{Embs}(D)\leftarrow$ the set of all sort embeddings in $D$\;
$W\leftarrow$ the set of all one-letter words over $\msym{Embs}(D)$\;
$\msym{rel}\leftarrow$ the least equivalence on $W$\;
\nl $\msym{rel}\leftarrow \msym{mergeEquivClassesBy}\;\msym{diagRule}\;\msym{rel}$\;
$\msym{rel}\leftarrow \msym{taggedValsToEquivClasses}\;\msym{rel}$\;
$W_2\leftarrow$ the set of all $\simeq$-admissible two-letter words over $\msym{Embs}(D)$\;
\ForEach{$\omega\in W_2$}{
  \ForEach{$\msym{eqcl}\in \msym{rel}$}{
    $\upsilon\leftarrow$ the first word in $\msym{eqcl}$\;
    \If{$dom(\omega)= dom(\upsilon)$ {\bf and} $cod(\omega)=
      cod(\upsilon)$}{
      add $\omega$ to $\msym{eqcl}$\;
      \KwSty{break}\;
    }
  }
}
\return{\sym{rel}};
\end{algorithm}

Computing $\sim$ (line \algline{7} in Algorithm
\ref{alg:ensuresAmalg}) boils down to applying the \rulename{Diag}
rule of cell calculus to $\msym{Embs}(D)$ and closing the resulting
relation w.r.t \rulename{Cong1} rule (see
Sect. \ref{impl:amalg:opt}). This is done in function \sym{sim}
presented in Algorithm \ref{alg:sim}. \sym{congruenceClosure} function
from line \algline{1} is described in Algorithm
\ref{alg:congClosure}. 
\begin{algorithm}[H]
\caption{\sym{sim} function\label{alg:sim}}
\dontprintsemicolon
\AlgData{$\bullet$ a diagram $D$\\$\bullet$ a set \sym{Embs} of sort embeddings represented as a list
  \sym{[DiagEmb]}} 
\AlgResult{the relation $\sim$ represented as \sym{EquivRel\;DiagEmb}} 
\BlankLine
$\msym{rel}\leftarrow$ the least equivalence on \sym{Embs}\;
$\msym{rel}\leftarrow
\msym{mergeEquivClassesBy}\;\msym{diagRule}\;\msym{rel}$\;
$\msym{check}\leftarrow\lambda (p, e).\lambda (q, d).$``$p = q$ and \sym{Embs}
 contains a pair $(p, ed)$''\;
$\msym{op}\leftarrow\lambda (p, e).\lambda (q, d).(p, ed)$\;
\Repeat{fixpoint on \sym{rel} is reached}{
  \nl$\msym{rel}\leftarrow
  \msym{congruenceClosure}\;\msym{check}\;\msym{op}\;\msym{rel}$\; 
}
$\msym{rel}\leftarrow \msym{taggedValsToEquivClasses}\;\msym{rel}$\;
\return{\sym{rel}}\;
\end{algorithm}
\begin{algorithm}[H]
\caption{\sym{congruenceClosure} function\label{alg:congClosure}}
\dontprintsemicolon
\AlgData{$\bullet$ \sym{check}$:\msym{E}\rightarrow\msym{E}\rightarrow
  \msym{Bool}$ --- a function that checks rule prerequisite\\$\bullet$ 
  \sym{op}$:\msym{E}\rightarrow\msym{E}\rightarrow\msaym{E}$ --- a
  function combining two elements\\$\bullet$ a relation \sym{rel} represented as
  \sym{EquivRelTagged\;E\;T}}
\AlgResult{a relation \sym{rel'} obtained from \sym{rel} by applying
the \rulename{Cong} rule once to each combination of elements in
\sym{rel}}
\BlankLine
\ForEach{pair $(\omega, t_\omega)\in \msym{rel}$}{
  \ForEach{pair $(\upsilon, t_\upsilon)\in \msym{rel}$ such that $t_\omega = t_\upsilon$}{
    \ForEach{pair $(\psi, t_\psi)\in \msym{rel}$ such that
      (\sym{check} $\omega$ $\psi$)}{
      \ForEach{pair $(\phi, t_\phi)\in \msym{rel}$ such that $t_\psi =
	t_\phi$ and (\sym{check} $\upsilon$ $\phi$)}{
	\lIf{a pair $(\msym{op}\;\omega\;\psi, t)\in \msym{rel}$}{
	  $t_{\omega\psi}\leftarrow t$\;
	}\lElse{
	  $t_{\omega\psi}\leftarrow \msym{Nothing}$\;
	}
	\lIf{a pair $(\msym{op}\;\upsilon\;\phi, t)\in \msym{rel}$}{
	  $t_{\upsilon\phi}\leftarrow t$\;
	}\lElse{
	  $t_{\upsilon\phi}\leftarrow \msym{Nothing}$\;
	}
	\If{$t_{\omega\psi}\neq \msym{Nothing}$ {\bf and}
	  $t_{\upsilon\phi}\neq \msym{Nothing}$}{
	  $\msym{rel}\leftarrow \msym{mergeEquivClasses}\;\msym{rel}\;t_{\omega\psi}\;t_{\upsilon\phi}$\;
	}
      }
    }
  }
}
\return{\sym{rel}}\;
\end{algorithm}

We take the first element of each equivalence class of $\sim$ to
be the canonical embedding that represents the whole equivalence
class. The translation from line \algline{8} in Algorithm
\ref{alg:ensuresAmalg} is therefore straightforward.
The check if $\msym{Adm}_\simeq$ is finite (line \algline{9} in Algorithm
\ref{alg:ensuresAmalg}) and the 
computation of this set (line \algline{10}) are actually performed
together by \sym{finiteAdm\_simeq} function (Algorithm
\ref{alg:finiteAdm}). $\epsilon$ denotes an empty word (i.e. empty
list). \sym{Nothing} doesn't mean non-termination --- it is a proper
return value. 
\begin{algorithm}[H]
\caption{\sym{finiteAdm\_simeq} function\label{alg:finiteAdm}}
\dontprintsemicolon
\AlgData{$\bullet$ a set \sym{Embs} of sort embeddings represented as a list
  \sym{[DiagEmb]}\\$\bullet$ a relation $\simeq$ that defines
  admissibility represented as \sym{EquivRel\;DiagSort}} 
\AlgResult{\sym{Just\;Adm_\simeq} if \sym{Adm_\simeq} is finite; \sym{Nothing}
  otherwise} 
\BlankLine
\nl\return{$\msym{embWords}\;\msym{Embs}\;(\simeq)\;\epsilon$}\;
\end{algorithm}
\begin{algorithm}[H]
\caption{\sym{embWords} function\label{alg:embWords}}
\dontprintsemicolon
\AlgData{$\bullet$ a set \sym{Embs} of sort embeddings represented as a list
  \sym{[DiagEmb]}\\$\bullet$ a relation $\simeq$ that defines
  admissibility represented as \sym{EquivRel\;DiagSort}\\$\bullet$ a word
  $\omega : \msym{DiagEmbWord}$} 
\AlgResult{$\msym{Just}\;W$ (where $W$ is the set of all the
  $\simeq$-admissible words over \sym{Embs}
  ending with $\omega$) if $W$ is finite; \sym{Nothing} otherwise} 
\BlankLine
\lIf{$\omega=\epsilon$}{
  $W\leftarrow$ an empty list\;
}\lElse{
  $W\leftarrow$ a singleton of $\omega$\;
}
\ForEach{$e\in \msym{Embs}$}{
  \If{$e\omega$ is $\simeq$-admissible}{
    \lIf{$e$ occurs in $\omega$}{
      \return{\sym{Nothing}}\;
    }\Else{
      $W_?'\leftarrow \msym{embWords}\;\msym{Embs}\;(\simeq)\;e\omega$\;
      \Switch{$W_?'$}{
	\Case{\sym{Nothing}}{\return{\sym{Nothing}}}
	\Case{$\msym{Just}\;W'$}{$W\leftarrow W\cup W'$}
      }
    }
  }
}
\return{$W$}\;
\end{algorithm}

The colimit thinness check from line \algline{11} in Algorithm
\ref{alg:ensuresAmalg} is probably the most complex piece of code in
the module. It is illustrated in Algorithm \ref{alg:colimitIsThin}.
\begin{algorithm}[H]
\caption{\sym{colimitIsThin} function\label{alg:colimitIsThin}}
\dontprintsemicolon
\AlgData{$\bullet$ a relation $\simeq$ represented as
  \sym{EquivRel\;DiagSort}\\$\bullet$ a set \sym{Embs} of sort
  embeddings (from diagram $D$) represented as list
  \sym{[DiagEmb]}\\$\bullet$ a relation 
  $\cong_0$ represented as \sym{EquivRel\;DiagEmbWord}}
\AlgResult{\sym{True} if the colimit of $D$ is thin, \sym{False} otherwise}
\BlankLine
\nl $\msym{Sorts_C}\leftarrow$ the list of first elements of each equivalence
class in $\simeq$\;
\nl $\simeq\leftarrow \msym{equivClassesToTaggedVals}\;\simeq$\;
\lForEach{$s\in \msym{Sorts_C}$}{
  $\msym{ordMap(s)} \leftarrow \emptyset$\;
}
\nl \lForEach{$e\in \msym{Embs}$}{
  $\msym{ordMap(dom_C(e))}\leftarrow \msym{ordMap(dom_C(e))}\cup\{cod_C(e)\}$\;
}
\nl\ForEach{pair $(s_1, s_2)$ of sorts from \sym{Sorts_C}}{
  $S_\geq(s_1, s_2)\leftarrow$ the set of colimit sorts $t$ such that
  $t\geq s_1$ and $t\geq s_2$\;
}
\nl\ForEach{pair $(s_1, s_2)$ of sorts from \sym{Sorts_C}}{
  $\msym{rel}\leftarrow$ the least equivalence on $S_\geq(s_1, s_2)$\;
  $\msym{mergeCond}\leftarrow\lambda s'.\lambda s''.S_\geq(s', s'')\neq\emptyset$ \;
  $\msym{rel}\leftarrow \msym{mergeEquivClassesBy}\;\msym{mergeCond}\;\msym{rel}$\;
  $B(s_1, s_2)\leftarrow \msym{taggedValsToEquivClasses}\;\msym{rel}$\;
}
\nl\While{\sym{ordMap} is not empty}{
  $s\leftarrow$ a sort $t\in \msym{Sorts_C}$ such that
  $\msym{ordMap(t)}=\emptyset$\;
  $\msym{ordMap}\leftarrow \msym{ordMap}$ with the mapping for $s$ removed\;
  \ForEach{$t\in \msym{Sorts_C}$ such that \sym{ordMap(t)} is defined}{
    $\msym{ordMap(t)}\leftarrow \msym{ordMap(t)}\setminus\{s\}$\;
  }
  \nl$\msym{Embs_s}\leftarrow$ all $e\in \msym{Embs}$ such that $dom_C(e) = s$\;
  $C\leftarrow$ an empty map\;
  \lForEach{pair $(d, e)$ of embeddings from \sym{Embs_s}}{
    $C(d, e)\leftarrow$ an empty list\;
  }
  \ForEach{pair $(d, e)$ of embeddings from \sym{Embs_s} such that
    $d\cong_0 e$}{
    $C(d, e)\leftarrow B(cod_C(d), cod_C(e))$\;
  }
  \ForEach{pair $(d, e)$ of embeddings from \sym{Embs_s} such that $dom(d)
    = dom(e)$}{
    \If{there exists $f\in \msym{Embs}$ such that $dom(f) = cod(d)$ and
      $cod(f) = cod(e)$}{
      $\beta\leftarrow$ the equivalence class from $B(cod_C(d),
      cod_C(e))$ that contains $cod_C(e)$\;
      add $\beta$ to $C(d, e)$ and to $C(e, d)$\;
    }
  }
  \Repeat{fixpoint of $C$ is reached}{
    \ForEach{tuple $(e_1, e_2, e_3)$ of embeddings from \sym{Embs_s}}{
      \ForEach{tuple $(\beta_{12}, \beta_{23}, \beta_{13})$ where
	$\beta_{12}\in C(e_1, e_2)$, $\beta_{23}\in C(e_2, e_3)$,
	$\beta_{13}\in B(e_1, e_3)$}{
	\If{$\beta_{12}\cap\beta_{23}\cap\beta_{13}\neq\emptyset$}{
	  add $\beta_{13}$ to $C(e_1, e_3)$ and to $C(e_3, e_1)$ if
	  it's not already there\;
	}
      }
    }
  }
  \nl\ForEach{pair $(d, e)$ of embeddings from \sym{Embs_s}}{
    \lIf{$B(cod_C(d),cod_C(e))\not\subseteq C(d, e)$}{
      \return{\sym{False}}\;
    }
  }
}
\return{\sym{True}}\;
\end{algorithm}

The set \sym{Sorts_C} in line \algline{1} is the set of colimit sorts; a
colimit sort is an equivalence class of $\simeq$, therefore an
element from each class is chosen to represent the colimit sort. For
embedding $e$, let $dom_C(e)$ and $cod_C(e)$ denote the colimit 
sorts --- i.e. the equivalence classes of $\simeq$ --- to which
(respectively) $dom(e)$ and $cod(e)$ belong. In line \algline{2} the
relation $\simeq$ is converted to 
\sym{EquivRelTagged} representation; each sort is tagged with the first
element of its equivalence class in $\simeq$. Note that the sorts used
for tags are the same sorts that represent colimit sorts in
\sym{Sorts_C}. The loop in line \algline{3} constructs \sym{ordMap} --- a map
that represents partial order on \sym{Sorts_C} by mapping $s$ to
the sorts immediately larger, i.e. such sorts $t$ that there exists an
embedding mapping sort from equivalence class of $s$ to a sort from
equivalence class of $t$. We use $\geq$ symbol to denote
the order imposed by \sym{ordMap}. The map $S_\geq$ constructed
in line \algline{4} maps each pair $(s_1, s_2)$ of colimit sorts to
the set of colimit sorts that are larger or equal than both $s_1$ and
$s_2$. The map $B$ created in line \algline{5} maps each pair $(s_1,
s_2)$ of colimit sorts to an equivalence on $S_\geq(s_1, s_2)$.

The colimit thinness check should answer the following question: are
all the cells over $D$ derivable in the cell calculus? 
We call a cell $(\omega,
\upsilon)$ an \nterm{$s$-cell} if the equivalence class of
$dom(\omega)$ and $dom(\upsilon)$ is $s$. The algorithm relies on the
observaion that for any
$s$-cell $c$ all the derivations of $c$ involve only such $t$-cells
for which $t\geq s$. The loop in line \algline{6} is the implementation
of an induction on the colimit sorts: assuming that for all
$t>s$ all $t$-cells belong to $\cong$, check that all $s$-cells belong
to $\cong$. 
The set $Embs_s$ computed in line \algline{7} contains all the sort
embeddings whose domains are in $s$. The map $C$ (computed in the
subsequent lines) maps a pair $(d, e)$ of embeddings from $Embs_s$ to
a set of cells 
where first path starts with $d$ and the second starts with
$e$. Roughly, $C$ can be thought of as a set of cells derivable in the
cell calculus, while $B$ represents all the cells in $D$; therefore
the check in line \algline{8} answers the question of colimit
thinness. 
For a thorough description of this algorithm and a proof of its
correctness please see the extended version of \cite{khtsm1}.

As mentioned above, $\msym{CanonicalEmbs}(D)$ (line \algline{12} in
Algorithm \ref{alg:ensuresAmalg}) is just a list consisting of the
first elements from each equivalence class of $\sim$. $Adm^C_\simeq$
in line \algline{13} is then computed using Algorithm
\ref{alg:finiteAdm} --- we ensured that $Adm_\simeq$ is finite, so
$Adm^C_\simeq$ must be finite as well. Generating $\cong$ (line
\algline{14}) is slightly more complicated than in case of
previous relations, since the rules of cell calculus (presented in
Sect. \ref{staticAnalysis:cellcalc}) and the rule \rulename{CompDiag}
(Sect. \ref{impl:amalg:opt}) need to be represented somehow in order
to use them in our framework for constructing equivalences. Rules
\rulename{Refl}, \rulename{Symm} and \rulename{Trans} hold
automatically in the construction presented. Rule \rulename{CompDiag}
is simple, since it does not require refering to 
existing relation; hence closures w.r.t. those rules can be computed
in one pass using \sym{mergeEquivClassesBy}. Closures for 
\rulename{Cong} and \rulename{Lc} are represented by functions
\sym{congruenceClosure} and \sym{leftCancellableClosure}
respectively, illustrated in Algorithms \ref{alg:congClosure} and
\ref{alg:lcClosure}. Note, that despite of function names, in order to
obtain actual closure one needs to subsequently apply
\sym{congruenceClosure} and \sym{leftCancellableClosure} until a
fixpoint is reached.
\begin{algorithm}[H]
\caption{\sym{leftCancellableClosure} function\label{alg:lcClosure}}
\dontprintsemicolon
\AlgData{a relation \sym{rel} represented as
  \sym{EquivRelTagged\;DiagEmbWord\;DiagEmbWord}}
\AlgResult{a relation \sym{rel'} obtained from \sym{rel} by applying
  the \rulename{Lc} rule once to each common prefix of each pair of
  words in \sym{rel}} 
\BlankLine
\ForEach{pair $(\omega, t_\omega)\in \msym{rel}$}{
  \ForEach{pair $(\upsilon, t_\upsilon)\in \msym{rel}$ such that
    $t_\upsilon=t_\omega$}{
    \Repeat{$e_\omega\neq e_\upsilon$ {\bf or} $\omega$ is empty
      {\bf or} $\upsilon$ is empty}{ 
      $e_\omega\leftarrow$ the first letter of $\omega$\;
      $\omega\leftarrow\omega$ with first letter removed\;
      $e_\upsilon\leftarrow$ the first letter of $\upsilon$\;
      $\upsilon\leftarrow\upsilon$ with first letter removed\;
      \If{$e_\omega = e_\upsilon$ {\bf and} $\omega$ is not empty
	{\bf and} $\upsilon$ is not empty}{
	$t_\omega\leftarrow t$ where $(\omega, t)\in \msym{rel}$\;
	$t_\upsilon\leftarrow t$ where $(\upsilon, t)\in \msym{rel}$\;
	$\msym{rel}\leftarrow \msym{mergeEquivClasses}\;\msym{rel}\;t_\omega\;t_\upsilon$\;
      }
    }
  }
}
\return{\sym{rel}}\;
\end{algorithm}

Now the \sym{cong} function used to compute $\cong$ may be defined as
shown in Algorithm \ref{alg:cong}. The $\cong^R$ from line
\algline{11} is computed using the same function --- only the set of
words passed to it is restricted to loopless words over canonical
embeddings. 
\begin{algorithm}[H]
\caption{\sym{cong} function\label{alg:cong}}
\dontprintsemicolon
\AlgData{$\bullet$ a set $W$ of embedding words, represented as
  list \sym{[DiagEmbWord]}\\$\bullet$ a relation $\simeq$ represented as
  \sym{EquivRel\;DiagSort}\\$\bullet$ a relation $\sim$ represented as
  \sym{EquivRel\;DiagEmb} (for \sym{compDiagRule})} 
\AlgResult{the least equivalence on $W$ satisfying all the rules of
  cell calculus w.r.t. $\simeq$} 
\BlankLine
$\msym{rel}\leftarrow$ the least equivalence on $W$\;
$\msym{rel}\leftarrow \msym{mergeEquivClassesBy}\;\msym{compDiagRule}\;\msym{rel}$\;
$\msym{check}\leftarrow\lambda \omega.\lambda
\upsilon.cod(\omega)\simeq dom(\upsilon)$\;
$\msym{op}\leftarrow\lambda \omega.\lambda \upsilon.\upsilon\omega$\;
\Repeat{fixpoint on \sym{rel} is reached}{
  $\msym{rel}\leftarrow \msym{congruenceClosure}\;\msym{check}\;\msym{op}\;\msym{rel}$\;
  $\msym{rel}\leftarrow \msym{leftCancellableClosure}\;\msym{rel}$\;
}
$\msym{rel}\leftarrow \msym{taggedValsToEquivClasses}\;\msym{rel}$\;
\return{\sym{rel}}\;
\end{algorithm}



% ====================================================================
% SUMMARY


\chapter{Summary}

Parsing and static analysis of architectural specifications and
amalgamability analysis for \CASL\ has been implemented in \Hets\ and 
presented in this thesis. While parsing and institution-independent
static analysis exactly follow the formal definitions presented in
\cite{refman}, it is the amalgamability analysis that turns
out most challenging.

Although the algorithms implemented perform in a satisfactory manner
for most of real-world specifications, their efficiency turns
out to be low in certain cases and optimisations should be
pursued. Even though in general the problems are PSPACE hard, it should
be possible to achieve significant performance improvements. In
particular \sym{congruenceClosure} and \sym{colimitIsThin} functions suffer
from high time-complexity. A variation of Nelson-Oppen congruence
closure algorithm \cite{no1} might be considered in the first
case. 

The comparison of architecture specfication analysis support in CATS
and \Hets\ 
revealed that while CATS delivers more consistent time performance it
sometimes fails to detect lack of amalgamability. E.g. although
the sample 
specification \ref{samp:failAmalg} is incorrect --- because of the problem
with embeddings described in \ref{staticAnalysis:cellcalc} --- CATS does
not issue any warning or error. Also the error messages given by
\Hets\ are more meaningful, as they specify which elements of
signatures do not ensure amalgamability. 

Further work on the architectural specification support in \Hets\
might include 
generating proof obligations when the amalgamability conditions cannot
be verified statically. This would also require implementing a
verification calculus for architectural specifications \cite{hoff1},
which is a complex area with several open research problems. 

\paragraph{Acknowledgments} I would like to express my gratitude to
Andrzej Tarlecki, who has introduced me to the subject of \CASL\ and
architectural specifications and has supervised this thesis. I would
also like to thank Till Mossakowski and Piotr Hoffman for their
assistance while implementing the analysis of architectural
specifications in \Hets.


% ====================================================================
% APPENDIX


\appendix

\chapter{Sample Specifications}

\section{Failing Sharing Check}
\begin{EXAMPLE}
\I\SPEC \NAMEDEFN{Spec\_1} = \SORT $s$
\I\SPEC \NAMEDEFN{Spec\_2} = \SORTS $s$, $t$
\\
\I\ARCHSPEC \NAMEDEFN{Arch\_Spec\_1} =
\begin{ITEMS}
\I\UNITS \[
  U : \NAMEDEFN{Spec\_1};\\
  F : \NAMEDEFN{Spec\_1} \TOTAL \NAMEDEFN{Spec\_2};\\
  G : \NAMEDEFN{Spec\_1} \TOTAL \NAMEDEFN{Spec\_2};
\]
\I\RESULT F[U] \AND G[U]
\end{ITEMS}
\I\END 
\end{EXAMPLE} 
\\\\
In the result of specification \NAMEDEFN{Arch\_Spec\_1} the sorts $t$
from F[U] and G[U] are identified although they might have different
interpretations.

\section{Using \rulename{Lc} Rule}
\begin{EXAMPLE}
\I\SPEC \NAMEDEFN{Spec\_1} = \SORTS $s < u$; $t<u$
\I\SPEC \NAMEDEFN{Spec\_2} = \SORTS $s < t$; $s < u$; $t<u$
\\
\I\ARCHSPEC \NAMEDEFN{Arch\_Spec\_1} =
\begin{ITEMS}
\I\UNITS \[
  U : \NAMEDEFN{Spec\_1};\\
  F : \NAMEDEFN{Spec\_1} \TOTAL \NAMEDEFN{Spec\_2};\\
  G : \NAMEDEFN{Spec\_1} \TOTAL \NAMEDEFN{Spec\_2};
\]
\I\RESULT F[U] \AND G[U]
\end{ITEMS}
\I\END 
\end{EXAMPLE} 
\\\\
Here the \rulename{Lc} rule of cell calculus is required to ensure
that $s<t$ in F[U] and $s<t$ in G[U] are equal.

\section{Failing Cell Condition}
\label{samp:failAmalg}
\begin{EXAMPLE}
\I\SPEC \NAMEDEFN{Spec\_1} = \SORTS $s$, $t$, $u$
\I\SPEC \NAMEDEFN{Spec\_2} = \SORTS $s < t$, $u$
\I\SPEC \NAMEDEFN{Spec\_3} = \SORTS $s < u$, $t < u$
\\
\I\ARCHSPEC \NAMEDEFN{Arch\_Spec\_1} =
\begin{ITEMS}
\I\UNITS \[
  U : \NAMEDEFN{Spec\_1};\\
  F : \NAMEDEFN{Spec\_1} \TOTAL \NAMEDEFN{Spec\_2};\\
  G : \NAMEDEFN{Spec\_1} \TOTAL \NAMEDEFN{Spec\_3};
\]
\I\RESULT F[U] \AND G[U]
\end{ITEMS}
\I\END 
\end{EXAMPLE} 
\\\\
This example (with more meaningful sort names) has been discussed in
\ref{staticAnalysis:cellcalc}. The embedding paths $s < u$ and $s < t
< u$ in the amalgamation are not guaranteed to commute.



\chapter{CD-ROM Contents}

The attached CD-ROM contains the source code of \Hets. This is the
version that was obtained from CVS at the moment of finishing this thesis.

\section{Directories}
\begin{itemize}
\item \path{HetCATS} -- the source code of \Hets,
\item \path{uni} -- the source code of UniForm Workbench that contains
  libraries required by \Hets.
\end{itemize}


\section{Compiling \Hets}
Current version of GHC\footnote{See
  \webaddr{http://www.haskell.org/ghc}} is required in order to
  compile \Hets. \Hets\ is known to compile on Linux and Solaris.

\begin{enumerate}
\item When in directory containing \path{uni} and \path{HetCATS}
  directories, enter the following commands:
  \begin{verbatim}
cd uni
./configure --enable-Het
make boot
make packages
cd ../HetCATS
  \end{verbatim}
\item Edit the \path{Makefile} in \path{HetCATS} directory: update the
  variable {\tt LINUX\_IMPORTS} to match the \path{imports} path of
  your GHC installation (usually \path{/usr/lib/ghc-*/imports}).
\item Enter {\tt make} in order to build \Hets.
\end{enumerate}


\section{Running \Hets}
In order to run \Hets\ enter the directory containing the executable
file \path{hets} and enter:
\begin{verbatim}
./hets
\end{verbatim}
This will display a quick reference of \Hets\ command-line
commands. The command flag \verb!--casl-amalg! gives the user control
over \CASL\ amalgamability checking algorithms. Its syntax is
\begin{verbatim}
--casl-amalg=<analysis option list>
\end{verbatim}
where {\tt<analysis option list>} is a comma-separated list of zero or
more of following options:
\begin{itemize}
\item {\tt sharing} --- perform sharing analysis,
\item {\tt cell} --- analyse the cell condition (both general and
  restricted); this option implies {\tt sharing},
\item {\tt colimit-thinness} --- check colimit thinness; this option
  implies {\tt sharing}.
\end{itemize}
For example, specifying \verb!--casl-amalg=colimit-thinness,cell! causes
colimit thinness and cell condition checks to be performed during
amalgamability analysis; sharing condition check is a prerequisite for
these, hence is performed as well. \verb!--casl-amalg=!\
(i.e. specifying empty list of options) turns the
amalgamability analysis for \CASL\ off completely. If not specified,
\verb!--casl-amalg=cell! is used by default.


\section{Accessing \Hets\ CVS Repository}

Should you require the up-to-date version of \Hets, it can be acquired
from the CVS repository at Bremen. Please follow the steps below:
\begin{enumerate}
\item set the {\tt CVSROOT} environment variable to 
\begin{verbatim}
:pserver:cvsread@cvs-agbkb.informatik.uni-bremen.de:/repository
\end{verbatim}
\item enter the command 
\begin{verbatim}
cvs login
\end{verbatim}
 and when prompted for password enter {\tt cvsread}
\item enter the command 
\begin{verbatim}
cvs export -r stable uni
\end{verbatim} 
this will download the sources of UniForm Workbench
\item enter the command 
\begin{verbatim}
cvs export HetCATS
\end{verbatim}
this will download the sources of \Hets
\end{enumerate}




\begin{thebibliography}{99}
\addcontentsline{toc}{chapter}{Bibliography}

\bibitem[GB92]{gb1} J. A. Goguen and R. M. Burstall. Institutions:
  Abstract Model Theory for Specification and
  Programming. \textit{J. ACM} 39(1), pp. 95-146, 1992
\bibitem[Hets04]{moss2} Till Mossakowski. \Hets\ User Guide, 2004
\bibitem[HM96]{hm1} Graham Hutton and Erik Meijer. Monadic Parser
  Combinators. Technical report NOTTCS-TR-96-4, Department of Computer
  Science, University of Nottingham, 1996
\bibitem[Hoff01]{hoff1} Piotr Hoffman. Verifying Architectural
  Specifications. In Maura Cerioli, Gianna Reggio (eds.): \textit{Recent
  Trends in Algebraic Development Techniques, 15th Intnl. Workshop,
  WADT 2001}. LNCS 2267, pp. 152-175, Springer 2001 
\bibitem[Klin00]{klin1} Bartosz Klin. Implementacja semantyki
  statycznej specyfikacji architekturalnych w formalizmie \CASL,
  Masters thesis, 2000 (in Polish)
\bibitem[KHTSM01]{khtsm1} Bartek Klin, Piotr Hoffman, Andrzej
  Tarlecki, Lutz Schr\"oder and Till Mossakowski. Checking
  Amalgamability Conditions for \CASL\ Architectural 
  Specifications. In Jiri Sgall, Ales Pultr, Petr Kolman (eds.):
  \textit{Mathematical Foundations of Computer Science, 26th Intnl. Symp.,
  MFCS 2001}. LNCS 2136, pp. 451-463, Springer 2001
\bibitem[Moss01]{moss1} Till Mossakowski. Implementing logics:
  from genericity to heterogeneity. Technical report
\bibitem[NO80]{no1} Greg Nelson and Derek C. Oppen. Fast Decision
  Procedures Based on Congruence Closure, \textit{J. ACM} 27(2),
  pp. 356-364, 1980 
\bibitem[RM04]{refman} Michel Bidoit and Peter D. Moses. \CASL\
  Reference Manual.LNCS Vol. 2960 (IFIP Series), Springer 2004 
\bibitem[SMTKH01]{smtkh1} Lutz Schr\"oder, Till Mossakowski, Andrzej Tarlecki,
  Bartek Klin and Piotr Hoffman. Semantics of Architectural
  Specifications in \CASL. In Heinrich Hussmann (ed.):
  \textit{Fundamental Approaches to Software Engineering, 4th
  Intnl. Conf., FASE 2001}. LNCS 2029, pp. 253-268, Springer 2001   
\bibitem[UM04]{userman} CoFI (The Common Framework Initiative). \CASL\
  User Manual. LNCS Vol. 2900 (IFIP Series), Springer 2004
\end{thebibliography}

\end{document}


%%% Local Variables: 
%%% mode: latex
%%% TeX-master: t
%%% coding: latin-2
%%% End: 
